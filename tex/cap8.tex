\chapter{Limites da Lógica Simbólica}

Vimos até agora as grandes potencialidades da lógica simbólica na resolução de validade de argumentos, consistência de um conjunto de sentenças, satisfatibilidade... Porém, a lógica simbólica têm limites para o que pode representar ou resolver. Observamos no capítulo anterior a conclusão de Alan Turing de que a Lógica de Primeira Ordem não é decidível (ou seja, não existe um procedimento confiável que nos permite dizer se qualquer fórmula é satisfatível ou não). Veremos nesse capítulo uma outra conclusão ainda mais intrigante, a qual chegou o matemático austríaco Kürt Godel.