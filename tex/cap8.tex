\chapter{A Incompletude da Matemática}

Vimos até agora as grandes potencialidades da lógica simbólica na resolução de validade de argumentos, consistência de um conjunto de sentenças, satisfatibilidade... Porém, a lógica simbólica têm limites para o que pode representar ou resolver. Observamos no capítulo anterior a conclusão de Alan Turing de que a Lógica de Primeira Ordem não é decidível (ou seja, não existe um procedimento confiável que nos permite dizer se qualquer fórmula é satisfatível ou não). Veremos nesse capítulo uma outra conclusão ainda mais intrigante, a qual chegou o matemático austríaco Kurt Gödel: na matemática, há enunciados verdadeiros para os quais não há prova.

\section{O Programa de Hilbert}

No fim do século XIX, os matemáticos estavam em busca de fundamentos para a matemática, ou seja, um conjunto contendo as definições mais elementares da matemática, que naquele momento, era tida como a Aritmética. Alguns matemáticos (incluindo Gottlob Frege, o fundador da Lógica de Primeira Ordem) tentaram definir tais fundamentos, mas foram pegos de surpresa com a aparição de paradoxos, que causou o questionamento dos métodos e lógica usada pela matemática. Esse período foi conhecido como a \textit{crise dos fundamentos da matemática}.

Em resposta, o alemão David Hilbert propôs uma solução bastante ambiciosa: representar cada uma das teorias da matemática por um conjunto finito de axiomas (leis básicas), e provar que esse conjunto é consistente usando a lógica de primeira ordem. Ao fazermos isso, \textbf{poderíamos então provar qualquer enunciado verdadeiro sobre essa teoria usando os axiomas}.

Conjuntos desse tipo são ditos \textbf{teorias axiomáticas}. Por exemplo, em 1889, Giuseppe Peano propôs uma assinatura e uma teoria axiomática para a Aritmética consistindo de 7 leis básicas:

\begin{table}[h]
    \centering
    \begin{tabularx}{\textwidth}{c X}
        \begin{structure}
            {$\mathbb{N}$}
            {}
            {$<$$(-,-)$}
            {$0$}
            {sucessor$(-)$\\$+(-,-)$\\$\times(-,-)$}
        \end{structure}
    &
        \begin{tabular}{l}
            1 símbolo de destaque: $0$;
            \\1 símbolo de relação binária: $<$;
            \\1 símbolo de função unária: $s$;
            \\2 símbolos de função binária: $+$, $\times$
        \end{tabular}
    \end{tabularx}
\end{table}

\begin{enumerate}
    \item \textbf{0 não é sucessor de nenhum número.}
    \\ $\neg\exists x(0 = s(x))$
    \item \textbf{A função sucessor é injetora.}
    \\ $\forall x\forall y((s(x) = s(y)) \rightarrow x = y)$
    \item \textbf{Lei da Indução Matemática}
    \\ $(P(0) \land \forall x(P(x) \rightarrow P(s(x)))) \rightarrow \forall y P(y)$
    \item \textbf{Lei da Recursividade da Adição}
    \\ $\forall x\forall y((x + s(y)) = s(x + y))$
    \item \textbf{0 é elemento neutro da adição.}
    \\ $\forall x((x + 0) = x)$
    \item \textbf{Lei da Recursividade da Multiplicação}
    \\ $\forall x\forall y((x \times s(y)) = (x \times y) + x)$
    \item \textbf{A multiplicação de qualquer número por 0 resulta em 0.}
    \\ $\forall x((x \times 0) = 0)$
\end{enumerate}

Podemos provar vários enunciados a partir desses axiomas, e então usá-los para provar mais enunciados. Por exemplo, podemos provar que 1 é elemento neutro da multiplicação usando os axiomas 5, 6 e 7:
\[\{\forall x((x + 0) = x), \forall x\forall y((x \times s(y)) = (x \times y) + x), \forall x((x \times 0) = 0)\} \vdash \forall x((x \times s(0)) = x)\]

E essa é a proposta de Hilbert. Uma vez que isso possa ser feito, teremos um sistema que é \textbf{completo} (todo enunciado verdadeiro é demonstrável a partir dos axiomas) e \textbf{correto} (todo enunciado que é demonstrável a partir dos axiomas é verdadeiro).
\section{O Teorema da Incompletude}
Contudo, em 1931, Kurt Gödel deu uma ``sentença de morte'' ao programa. Ele usou a Aritmética de Peano para provar seu primeiro \textbf{Teorema da Incompletude}:

\begin{theorem}{Primeiro Teorema da Incompletude}
Qualquer teoria axiomática que se proponha a formalizar a Aritmética não pode ser completa e correta ao mesmo tempo.
\end{theorem}
O teorema implica que, se supormos que tudo que demonstramos é verdadeiro (corretude), então há enunciados verdadeiros na matemática que não possuem prova (não completude). Por outro lado, se supormos que podemos provar qualquer enunciado verdadeiro (completude), então também podemos provar enunciados falsos (não corretude). 

\subsection{A Estratégia de Gödel}
A ideia inusitada de Gödel foi usar a Aritmética para falar de si própria: ele propôs uma maneira de escrever quaisquer sentenças da Aritmética sob a forma de um natural. Para isso, ele atribuiu um número primo a todo símbolo da assinatura de Peano e da lógica de primeira ordem. Por exemplo:

\begin{table}[h]
    \centering
    \begin{tabular}{c c c c c c c c c c c c c c c c c}
        $0$ & $<$ & $s$ & $+$ & $\times$ & $\forall$ & $\exists$ & $\neg$ & $\land$ & $\lor$ & $\rightarrow$ & $x$ & y & $=$ & ( & ) &...
        \\
        2 & 3 & 5 & 7 & 11 & 13 & 17 & 19 & 23 & 29 & 31 & 37 & 41 & 43 & 47 & 53
    \end{tabular}    
\end{table}

Assim, ele propôs escrever quaisquer enunciados da Aritmética, verdadeiros ou falsos, sob a forma de produtos de números primos, usando a codificação acima como os seus expoentes. Por exemplo:
\[\neg \exists x(0 = s(x)) \equiv 2^{19} \times 3^{17} \times 5^{37} \times 7^{47} \times 11^{2} \times 13^{43} \times 17^5 \times 19^{47} \times 23^{37} \times 29^{53} \times 31^{53}\]
\[\forall x((x + 0) = 0) \equiv 2^{13} \times 3^{47} \times 5^{47} \times 7^{37} \times 11^{7} \times 13^{2} \times 17^{53} \times 19^{43} \times 23^{2} \times 29^{53}\]

O Teorema Fundamental da Aritmética garante que cada número possui uma fatoração prima única, então a codificação de Gödel é válida.

\subsubsection{O Paradoxo do Mentiroso}
O Paradoxo do Mentiroso é a seguinte sentença: ``Eu não sou verdadeira''. É um paradoxo pois, se supormos verdadeira, ela se afirma falsa e, se supormos falsa, ela se afirma verdadeira. Gödel tomou uma variante do paradoxo:
\begin{center}
    \textbf{Eu não sou demonstrável a partir dos axiomas.}
\end{center}
A sentença de Gödel não é um paradoxo, mas serviu para seu propósito. Gödel conseguiu codificar a sentença como um natural, usando a codificação anterior, implicando que ela é formalmente um enunciado da Aritmética. Tal feito é intrigante, pois:

\begin{itemize}
    \item Supondo que a sentença é falsa, temos que ela é demonstrável a partir dos axiomas. Supondo que a Aritmética é correta, a sentença deve ser verdadeira, o que é uma contradição. Desse modo, a suposição de que ela é falsa não pode ser verdade, e portanto, a sentença deve ser verdadeira.
    \item Uma vez que é verdadeira, ela se afirma não demonstrável a partir dos axiomas. Assim, há uma sentença verdadeira na matemática que não podemos provar verdadeira, implicando que a Aritmética não é completa.
    \item Podemos fazer o mesmo raciocínio com a suposição de que a Aritmética é completa. Sendo assim, a sentença não pode ser verdadeira. Desse modo, há uma sentença demonstrável na Aritmética, mas que é falsa, implicando que a Aritmética não é correta.
\end{itemize}

Assim, a Aritmética não pode ser correta e completa ao mesmo tempo.