\section{Subestruturas}

Como saber se uma estrutura $A$ é subestrutura de uma estrutura $B$? Se $A$ e $B$ forem simplesmente conjuntos, basta saber se todos os elementos de $A$ também são elementos de $B$. Mas, considerando os outros componentes das estruturas $A$ e $B$ (relações, destaques e funções), é necessário verificar se esses componentes possuem uma relação entre si que justifique dizer que $A$ está contida em $B$ como estrutura.

Para definir matematicamente esse possível relacionamento, tomamos emprestado da álgebra a noção de \textbf{homomorfismo}: uma função que preserva propriedades.

\begin{definition}{Homomorfismo}
    Sejam $A$ e $B$ estruturas de uma mesma assinatura $L$. Uma função $h: dom(A) \mapsto dom(B)$ é dita homomorfismo de $A$ para $B$ se as condições seguintes forem satisfeitas.
    \begin{enumerate}
        \item Para todo símbolo de constante $c$ de $L$, $h(c^A) = c^B$;
        \item Para todo símbolo de relação $n$-ária $R$ de $L$ e toda $n$-upla $(a_1,...,a_n)$ de elementos de $A$, $(a_1,...a_n) \in R^A \rightarrow (h(a_1),...,h(a_n)) \in R^B$;
        \item Para todo símbolo de função $n$-ária $f$ de $L$ e toda $n$-upla $(a_1,...,a_n)$ de elementos de $A$, $h(f^A(a_1,...a_n)) = f^B(h(a_1),...,h(a_n))$.  
    \end{enumerate}
\end{definition}

Para ilustrar esse conceito, tomemos duas estruturas $A$ e $B$:

\begin{table}[h]
    \centering
    \begin{tabular}{c c}
        $\mathbf{A}$ & $\mathbf{B}$ \\
        \begin{structure}
            {}
            {$0$ $1$ $3$ $5$}
            {$R_1(-,-)$\\$R_2(-)$}
            {$1$ $3$}
            {$f(-)$}
        \end{structure}
        &
        \begin{structure}
            {}
            {$0$ $1$ $2$ $3$ $5$}
            {$R_3(-,-)$\\$R_4(-)$}
            {$1$ $2$}
            {$g(-)$}
        \end{structure} 
    \end{tabular}
\end{table}

Suponha que:
\begin{description}
    \item $R_1 = \{(0,3), (1,3), (3,5), (5, 3)\}$ \quad|\quad $R_3 = \{(0,3), (1,2), (3,5), (2, 3), (3,2), (3,3)\}$
    \item $R_2 = \{0,1,5\}$ \quad|\quad $R_4 = \{0,1,2,3,5\}$
    \item $f(0) = 1, f(1) = 1, f(3) = 2, f(5) = 3$ \quad|\quad $g(0) = 0, g(1) = 1, g(2) = 2, g(3) = 3, g(5) = 5$
\end{description}
Seja $h: dom(A) \mapsto dom(B)$ uma função entre as duas estruturas, definida da seguinte forma:
\begin{center}
    $h(0) = 1$ \\
    $h(1) = 1$ \\
    $h(3) = 2$ \\
    $h(5) = 3$
\end{center}
$h$ é um homomorfismo de $A$ para $B$? Vamos verificar cada condição:
\begin{enumerate}
    \item A 1ª condição diz que os destaques de $A$ são mapeados para destaques de $B$. Notamos que $h(1) = 1$ e $h(3) = 2$. Assim, a 1ª condição é satisfeita e dizemos que $h$ \textbf{preserva destaques}.
    \item A 2ª condição diz que se uma tupla de elementos se relaciona em $A$, então a tupla contendo os mapeamentos desses elementos se relaciona em $B$. Analisando as relações:
    \begin{description}
        \item[$R_1$:] $(0,3) \mapsto (h(0),h(3)) = (1,2) \in R_3$\\
        $(1,3) \mapsto (h(1),h(3)) = (1,2) \in R_3$ \\
        $(3,5) \mapsto (h(3),h(5)) = (2,3) \in R_3$ \\
        $(5,3) \mapsto (h(5),h(3)) = (3,2) \in R_3$
        \item[$R_2$:] $0 \mapsto h(0) = 1 \in R_4$ \\
        $1 \mapsto h(1) = 1 \in R_4$ \\
        $5 \mapsto h(5) = 3 \in R_4$
    \end{description}
    Assim, a 2ª condição é satisfeita e dizemos que $h$ \textbf{preserva relações}.
    \item A 3ª condição diz que mapear a aplicação de uma função em $A$ corresponde a mapear primeiro os argumentos e depois aplicar uma função em $B$. Analisando as funções:
    \begin{center}
        $h(f(0)) = g(h(0)) = g(1) = 1$ \\
        $h(f(1)) = g(h(1)) = g(1) = 1$ \\
        $h(f(3)) = g(h(3)) = g(2) = 2$ \\
        $h(f(5)) = g(h(5)) = g(3) = 3$
    \end{center}
    Assim, a 3ª condição é satisfeita e dizemos que $h$ \textbf{preserva funções}. Por preservar destaques, relações e funções, $h$ é um homomorfismo de $A$ para $B$.
\end{enumerate}
     
\subsection{Imersão}
Um homomorfismo $h: dom(A) \rightarrow dom(B)$ é dito \textbf{imersão} se:
\begin{itemize}
    \item $h$ é injetora;
    \item $h$ satisfaz uma versão mais forte da 2ª condição:
    \\ Para todo símbolo de relação $n$-ária $R$ de $L$ e toda $n$-upla $(a_1,...,a_n)$ de elementos de $A$, $(a_1,...a_n) \in R^A \leftrightarrow (h(a_1),...,h(a_n)) \in R^B$.
\end{itemize}
A função $h$ do exemplo anterior não é uma imersão, uma vez que, não só ela quebra a primeira condição (pois $h(0) = h(1) = 1$, implicando que $h$ não é injetora) como a segunda ($(5,5) \notin R_1$, mas $(h(5),h(5)) = (3,3) \in R_2$).
<<<<<<< HEAD
Além da imersão, existem outras variantes do homomorfismo:
=======
Além da imersão, existem outras variantes para o homomorfismo:
>>>>>>> 577f0e6b92bc77b1677a65462ddcc92c8313823c
\begin{itemize}
    \item Uma imersão sobrejetora é dita \textbf{isomorfismo}.
    \item Um homomorfismo $h: dom(A) \mapsto dom(A)$ é dito \textbf{endomorfismo} de $A$.
    \item Um isomorfismo $h: dom(A) \mapsto dom(A)$ é dito \textbf{automorfismo} de $A$.
\end{itemize}

Agora, podemos remeter ao problema inicial e definir então as condições para que uma estrutura $A$ seja subestrutura de uma estrutura $B$.
\begin{definition}{Subestrutura}
    Sejam $A$ e $B$ estruturas de mesma assinatura. Dizemos que $A$ é subestrutura de $B$ se:
    \begin{enumerate}
        \item $dom(A) \subseteq dom(B)$
        \item A função identidade $i: dom(A) \mapsto dom(B)$ $|$ $i(x) = x$ é uma imersão.
    \end{enumerate}
    A notação é $A \subseteq B$.
\end{definition}

\subsection{O Problema da Menor Subestrutura}

Seja $A$ a estrutura a seguir e $X = \{0,1,3\}$ um subconjunto do domínio de $A$:
\begin{center}
    \begin{structure}
        {}
        {$0$ $1$ $2$ $3$ $4$ $5$}
        {primo$(-)$\\$<$$(-,-)$}
        {$1$ $2$}
        {quadrado-mod-5$(-)$\\soma-mod-5$(-,-)$}
    \end{structure} 
\end{center}

Queremos construir uma subestrutura de $A$ que contenha o menor número de elementos em seu domínio e que contenha $X$. Estamos diante de um problema de otimização:

\begin{description}
    \item[Dada:] uma $L$-Estrutura $A$ e um conjunto $X \subseteq dom(A)$;
    \item[Pergunta-se:] qual a menor subestrutura $B$ de $A$ que contém $X$, ou seja, $B \subseteq A$ e $X \subseteq dom(B)$?   
\end{description}
A notação que usamos para $B$ é $\langle X \rangle_A$. Assim, $B$ deve conter os mesmos destaques, relações e funções que $A$ e deve conter $X$ em seu domínio. Além disso, precisamos adicionar elementos ao domínio de $B$ para que a definição de estrutura se mantenha consistente. 

Inicialmente, temos a seguinte estrutura:
\begin{center}
    \begin{structure}
        {}
        {$0$ $1$ $3$}
        {primo$(-)$\\$<$$(-,-)$}
        {$1$ $2$}
        {quadrado-mod-5$(-)$\\soma-mod-5$(-,-)$}
    \end{structure} 
\end{center}

<<<<<<< HEAD
Note que ela possui o destaque $2$, que não pertence ao domínio. Por definição, o conjunto de destaques é subconjunto do domínio, logo, devemos adicioná-lo a este:
=======
Note que ela possui o destaque $2$, que não pertence ao domínio. Por definição, o conjunto de destaques é subconjunto do domínio, logo, devemos adicionar o elemento ao domínio:
>>>>>>> 577f0e6b92bc77b1677a65462ddcc92c8313823c
\begin{center}
    \begin{tikzpicture}[modal]
        \node[world] (Dom) [align=left] {$0$ $1$ $2$ $3$};
    \end{tikzpicture} 
\end{center}

Note também que a função quadrado-mod-5 aplicada a $3$ retorna $4$, que não é um elemento do domínio. Por definição, o domínio é fechado sob as funções, assim, devemos adicionar $4$ ao domínio:

\begin{center}
    \begin{tikzpicture}[modal]
        \node[world] (Dom) [align=left] {$0$ $1$ $2$ $3$ $4$};
    \end{tikzpicture} 
\end{center}

Dessa forma, $\langle X \rangle_A$ é a estrutura com domínio $\{0,1,2,3,4\}$ e com os mesmos destaques, relações e funções que $A$. 

Podemos sintetizar o procedimento para construir $\langle X \rangle_A$ da seguinte forma:
\begin{enumerate}
<<<<<<< HEAD
    \item Inicialmente, adicione os destaques, funções e relações de $A$ e faça $dom(B) = X$.
    \item Adicione os destaques de $B$ ao domínio de $B$.
    \item Repita até que nenhum elemento novo seja adicionado:
    \begin{enumerate}
        \item Adicione os conjuntos imagens das funções de $B$ ao domínio de $B$.
    \end{enumerate}  
=======
    \item Inicialmente, adicionamos os destaques, funções e relações de $A$ e fazemos $dom(B) = X$.
    \item Adicione os destaques de $B$ ao domínio de $B$.
    \item Adicione os conjuntos imagens das funções de $B$ ao domínio de $B$.  
>>>>>>> 577f0e6b92bc77b1677a65462ddcc92c8313823c
\end{enumerate}

\subsection{Extensão de uma estrutura}

Quando $\langle \emptyset \rangle_A = A$, ou seja, a menor subestrutura de $A$ construída a partir do conjunto vazio como domínio é a própria $A$, todos os elementos de $A$ são alcançáveis a partir dos destaques e funções de $A$.

Isso não é verdade na estrutura do exemplo anterior, uma vez que $5$ é um elemento inalcançável a partir dos destaques e funções disponíveis e é considerado ``sem nome'' (é impossível representar 5 por meio de símbolos sobre essa estrutura). Dessa forma, podemos extender a estrutura $A$, adicionando ao seu conjunto de destaques os elementos inacessíveis de $A$. A estrutura resultante $A'$ é chamada de \textbf{extensão} de $A$ (e $A$ é dita \textbf{reduto} de $A'$).
    
A extensão da estrutura do exemplo anterior é, portanto:
\begin{center}
    \begin{structure}
        {}
        {$0$ $1$ $2$ $3$ $4$ $5$}
        {primo$(-)$\\$<$$(-,-)$}
        {$1$ $2$ $5$}
        {quadrado-mod-5$(-)$\\soma-mod-5$(-,-)$}
    \end{structure} 
\end{center}