\chapter{Problema da Satisfatibilidade}

\begin{description}
    \item[Dada:] uma fórmula $\varphi$;
    \item[Pergunta-se:] $\varphi$ é satisfatível? 
\end{description}

Retomamos o Problema da Satisfatibilidade (SAT), dessa vez sob a perspectiva da lógica de primeira ordem. Porém, temos uma peculiaridade envolvendo satisfatibilidade aqui. Em lógica proposicional, podemos avaliar se uma proposição é tautologia verificando se ela é verdadeira sob todas as valorações-verdade possíveis para a proposição. Essa maneira de força-bruta de resolver não funciona em lógica de predicados, uma vez que precisaríamos avaliar uma fórmula sob todas as interpretações em todas as estruturas (de mesma assinatura da fórmula), e existe um número infinito de estruturas. Assim, para determinar invalidez de uma fórmula, os algoritmos devem ser capazes de detectar que ela é falsa em um número finito de testes. Mas há casos em que isso simplesmente não é possível devido à natureza da fórmula. Desse modo, os algoritmos de primeira ordem conseguem decidir se uma fórmula é válida se ela de fato for válida, mas, caso não seja, eles podem nunca terminar. Alonzo Church e Alan Turing provaram em 1936 que um algoritmo que consegue decidir corretamente se qualquer fórmula é válida (ou insatisfatível) ou não não pode existir. Desse modo, classificamos a lógica de primeira ordem como \textbf{semi-decidível}.
\section{Método da Resolução}

A filosofia do método persiste: ser eficiente para certas entradas e eficiente em reconhecer essas entradas. Os conceitos como a regra da resolução e a forma normal conjuntiva continuam valendo, mas dessa vez, temos diferenças cruciais.
\subsection{O Problema da Unificação de Termos}
Se em uma cláusula $C_1$ há um literal $L$ e em outra cláusula $C_2$ há o literal $\neg L$, a regra da resolução determina que podemos criar uma cláusula nova -- o resolvente de $C_1$ e $C_2$, sendo a disjunção de todos os outros literais de $C_1$ e $C_2$. Porém, uma condição importante é que $L$ em $C_1$ seja idêntico a $L$ (de $\neg L$) em $C_2$.

Em lógica proposicional, esse problema é trivial pois avaliamos apenas variáveis. Mas em primeira ordem, caso dois literais não sejam idênticos, devemos ser capazes de detectar se eles \textbf{podem} ser idênticos, utilizando de substituição de variáveis. Por exemplo, podemos aplicar a regra em $R(a)$ e $\neg R(x)$ se tomarmos $[a\diagup x]$, mas o mesmo não é possível com $P(a)$ e $\neg P(b)$, pois não há substituição que torne os literais idênticos. De modo geral, precisamos de um procedimento que determine se dois literais podem ser idênticos.

\begin{description}
    \item[Dados:] dois termos $t_1$ e $t_2$;
    \item[Pergunta-se:] existe uma substituição de variáveis que torne $t_1$ e $t_2$ idênticos? 
\end{description}

Caso tal substituição exista, dizemos que $t_1$ e $t_2$ são termos \textbf{unificáveis} e a substituição é dita \textbf{unificadora}. Em 1930, o matemático Jacques Herbrand definiu um algoritmo simples que resolve o problema usando regras de transformação de um conjunto de equações. Uma \textbf{equação} é um par de termos $(s = t)$. Um \textbf{sistema de equações} $S$ é um multiconjunto de equações e uma substituição $\Theta$ é unificadora de $S$ se ela unifica todas as equações de $S$. O conjunto de todas as substituições unificadoras de $S$ é denotado por $U(S)$. Esse conjunto pode conter várias substituições possíveis, mas uma delas é a ``ótima''. Por exemplo, para os termos abaixo:
\[f(y))\quad\text{e}\quad f(g(z))\]
A substituição $\Theta_1 = [g(b)\diagup y, b\diagup z]$ unifica os termos, mas não é a única possível. $\Theta_2 = [g(z)\diagup y]$ também os unifica e é mais simples. A substituição mais simples que unifica dois termos dentre as possíveis é dita \textbf{unificadora mais geral}.

O algoritmo de Herbrand usa como base o conceito de \textbf{forma resolvida}: uma equação está na forma resolvida em um sistema $S$ se for da forma $v = t$ (variável = termo) e $v$ for uma variável resolvida, ou seja, $v$ não ocorre em $t$ e em nenhuma outra equação de $S$. Caso todas as equações de $S$ estejam na forma resolvida, o sistema é unificável e o algoritmo devolve a substituição unificadora mais geral de $S$.

\begin{definition}{Regras de Transformação}
    \begin{description}
        \item[Eliminação de Equações Triviais] $S \cup \{t \? t\} \Rightarrow S$
        \item[Decomposição de Termos] $S \cup \{f(t_1,...,t_n) \? f(s_1,...,s_n)\} \Rightarrow S \cup \{t_1 \? s_1,...,t_n \? s_n\}$
        \begin{itemize}
            \item $S \cup \{f(t_1,...,t_n) \? g(t_1,...,t_m)\}$, onde $f \neq g$ ou $n \neq m \Rightarrow$ \textit{falha}.
        \end{itemize}  
        \item[Eliminação de Variáveis] $S \cup \{x \? t\} \Rightarrow S[x\diagup t] \cup \{x \? t\}$, se $x$ não ocorre em $t$.
        \begin{itemize}
            \item Se $x$ ocorre em $t \Rightarrow$ \textit{falha}.
        \end{itemize}
    \end{description}
    \tcbsubtitle{Algoritmo de Herbrand}
    \begin{description}
        \item[Entrada:] um sistema de equações $S$;
        \item[Saída:] Se $S$ for unificável, a unificadora mais geral de $S$; caso contrário, $S$ não é unificável.
    \end{description}
    \begin{enumerate}
        \item Para cada equação $s$ de $S$:
        \begin{enumerate}
            \item [1.1] Se $s$ estiver na forma resolvida, passe para a próxima.
            \item [1.2] Caso contrário, aplique uma das regras de transformação, com prioridade decrescente de cima pra baixo, e vá para o início.
            \item [1.3] Se não for possível aplicar uma regra, retorne \textbf{não é unificável}.
        \end{enumerate}
        \item Caso todas as equações de $s$ estejam na forma resolvida, retorne a \textbf{unificadora mais geral} formada pelas equações presentes em $S$.
    \end{enumerate}
\end{definition}

Para ilustrar, vamos executar o algoritmo sobre o seguinte conjunto de equações $S$:
\[S = \{f(g(z), x) = f(y, x), f(y,x) = f(y,h(a)), f(g(z), x) = f(y, h(a))\}\]
\begin{center}
    \begin{tabular}{r l}
        \textbf{DT} & $\{g(z) \? y, x \? x, f(y,x) \? f(y,h(a)), f(g(z), x) \? f(y, h(a))\}$ \\
        \textbf{EV} & $\{\mathbf{y \? g(z)}, x \? x, f({\color{Orange}g(z)},x) \? f({\color{Orange}g(z)},h(a)), f(g(z), x) \? f({\color{Orange}g(z)}, h(a))\}$ \\
        \textbf{EET} & $\{\mathbf{y \? g(z)}, f(g(z),x) \? f(g(z),h(a)), f(g(z), x) \? f(g(z), h(a))\}$ \\
        \textbf{DT} & $\{\mathbf{y \? g(z)}, g(z) \? g(z), x \? h(a), f(g(z), x) \? f(g(z), h(a))\}$ \\
        \textbf{EET} & $\{\mathbf{y \? g(z)}, x \? h(a), f(g(z), x) \? f(g(z), h(a))\}$ \\
        \textbf{EV} & $\{\mathbf{y \? g(z)}, \mathbf{x \? h(a)}, f(g(z), {\color{Orange}h(a)}) \? f(g(z), h(a))\}$ \\
        \textbf{DT} & $\{\mathbf{y \? g(z)}, \mathbf{x \? h(a)}, g(z) \? g(z), h(a) \? h(a)\}$ \\
        \textbf{EET} & $\{\mathbf{y \? g(z)}, \mathbf{x \? h(a)}, h(a) \? h(a)\}$ \\
        \textbf{EET} & $\{\mathbf{y \? g(z)}, \mathbf{x \? h(a)}\}$ 
    \end{tabular}
\end{center}
$S$ é unificável e a unificadora mais geral de $S$ é $\Theta = [g(z)\diagup y, h(a) \diagup x]$. Isso significa que, para os termos serem idênticos, basta substituir $x$ por $h(a)$ e $y$ por $g(z)$. Vejamos agora um exemplo onde o algoritmo falha:
\[S' = \{g(f(x,x)) = g(f(h(a), g(b)))\}\]
\begin{center}
    \begin{tabular}{l l}
        \textbf{DT} & $\{f(x,x) \? f(h(a), g(b))\}$ \\
        \textbf{DT} & $\{x \? h(a), x \? g(b)\}$ \\
        \textbf{EV} & $\{\mathbf{x \? h(a)}, {\color{Orange}h(a)} \? g(b)\}$ \\
        & $\{\mathbf{x \? h(a)}, {\color{Red}h(a) \? g(b)}\}$
    \end{tabular}
\end{center}

Uma vez que $h \neq g$, o algoritmo falha ao tentar aplicar uma regra em $h(a) \? g(b)$. Desse modo, o sistema não é unificável. De fato: não existe nenhuma substituição de variáveis que torne $x = h(a)$ e $x = g(b)$ ao mesmo tempo. 

No contexto de resolução, portanto, usaremos o algoritmo de Herbrand sempre que tentarmos aplicar a regra da resolução em dois literais complementares. Caso sejam unificáveis, procedemos com o resolvente e \textbf{aplicamos a unificadora mais geral neste}. Caso contrário, não aplicaremos a regra.

\subsection{Remoção dos quantificadores}
Já concluímos um dos problemas referentes ao método da resolução, que se preocupava com a identicidade de literais. Agora, veremos o outro grande problema, sobre a sintaxe da entrada. As entradas do método da resolução devem estar na forma normal conjuntiva, que, como sabemos, é uma conjunção de cláusulas. Porém, as fórmulas da lógica de primeira ordem podem possuir quantificadores, que não estão presentes nessa definição. Dessa maneira, precisamos removê-los. A estratégia que usaremos será a seguinte: colocar todos os quantificadores no início da fórmula, e então removê-los.

\subsubsection{Forma Normal Prenex}
Uma fórmula $\varphi$ está na \textbf{forma normal prenex} (\textit{prefixed normal expression}) se, e somente se, ela tem o seguinte formato:
\[(Q_1 x_1Q_2 x_2...Q_n x_n)(\psi)\]
Onde $(Q_1 x_1Q_2 x_2...Q_n x_n)$ é uma parte com quantificadores e chamada de \textbf{prefixo} de $\varphi$ e $\psi$ é uma fórmula sem quantificadores chamada de \textbf{matriz} de $\varphi$.

\begin{theorem}{}
    Para toda fórmula $\varphi$, existe uma fórmula $\psi$ tal que:
    \begin{itemize}
        \item $\psi$ está na forma normal prenex;
        \item $\psi$ é logicamente equivalente a $\varphi$.
    \end{itemize}
    \tcbsubtitle{Prova (transformação para a FNP)}
    Para provar o teorema, faremos uma prova por construção mostrando um método que transforma qualquer fórmula $\varphi$ para outra $\psi$ logicamente equivalente e na forma normal prenex.
    \begin{enumerate}
        \item Elimine os conectivos $\rightarrow$ e $\leftrightarrow$.
        \item Aplique repetidamente, além das equivalências conhecidas $(\neg \neg L) \equiv L$ (lei da dupla negação), as leis de De Morgan e a propriedade distributiva, as equivalências a seguir, até que a fórmula esteja na forma normal prenex.
        \begin{itemize}
            \item $\neg \forall x (\omega) \equiv \exists x (\neg \omega)$
            \item $\neg \exists x (\omega) \equiv \forall x (\neg \omega)$
            \item $\forall x (\varphi(x)) \land \forall x (\psi(x)) \equiv \forall x (\varphi(x) \land \psi(x))$
            \item $\exists x (\varphi(x)) \lor \exists x (\psi(x)) \equiv \exists x (\varphi(x) \lor \psi(x))$
            \item $Q x (\varphi(x))\equiv Q y(\varphi(y))$, onde $Q \in \{\forall, \exists\}$
            \item $Q x (\varphi(x)) \land \psi \equiv Q x (\varphi(x) \land \psi)$, onde $Q \in \{\forall, \exists\}$ 
            \item $Q x (\varphi(x)) \lor \psi \equiv Q x (\varphi(x) \lor \psi)$, onde $Q \in \{\forall, \exists\}$ 
        \end{itemize}
    \end{enumerate}
\end{theorem}

\subsubsection{Forma Padrão de Skolem}

Uma fórmula $\varphi$ na forma normal prenex está na \textbf{forma padrão de Skolem} se, e somente se, o prefixo de $\varphi$ não contém quantificadores existenciais. Em 1920, o lógico-matemático Thoralf Skolem definiu um método para remoção de quantificadores existenciais, que acabou ganhando seu nome: o \textbf{método da skolemização}.

Suponha a seguinte estrutura $A$:
\begin{center}
    \begin{structure}
        {$\mathbb{N}$}
        {}
        {$<$$(-,-)$}
        {}
        {}    
    \end{structure}
\end{center}


A sentença $\varphi = \forall x \exists y R(x,y)$ é verdadeira em $A$, pois para todo natural $x$ existe um outro maior que ele, por exemplo, seu sucessor. Dessa forma, podemos retirar o existencial, substituindo a ocorrência de sua variável pela função sucessor aplicada a $x$, ou seja, $\varphi' = \forall x R(x,f(x))$.
\begin{center}
    \begin{structure}
        {$\mathbb{N}$}
        {}
        {$<$$(-,-)$}
        {}
        {sucessor$(-,-)$}    
    \end{structure}
\end{center}

$\varphi'$ é verdadeira nessa nova estrutura que é idêntica a $A$, adicionando-se a função $f$. A skolemização é, portanto, o processo de remover quantificadores existenciais, substituindo as ocorrências das variáveis desses quantificadores por funções aplicadas a variáveis universalmente quantificadas anteriores a esses existenciais.

\begin{theorem}{Teorema de Löwenheim-Skolem}
    Seja $\varphi$ uma fórmula na forma normal prenex sobre uma assinatura $L$. Seja $\psi$ a fórmula resultante da remoção de cada quantificador existencial que ocorre em $\varphi$ e cujas variáveis correspondentes são substituídas por um termo do tipo $f(x_1,...,x_n)$, onde $f$ é um novo símbolo de função e $x_1,...,x_n$ são variáveis universalmente quantificadas imediatamente anteriores a esse existencial. Então, se existe uma $L$-Estrutura $A$ que é modelo para $\varphi$, é possível construir uma $L'$-Estrutura $B$ que é modelo para $\psi$, acrescentando a $A$ uma interpretação para cada símbolo novo de função em $L'$.
\end{theorem}
Caso não haja quantificadores universais anteriores a um existencial, a variável correspondente deste é substituída por um novo símbolo de constante (função de aridade zero). Uma função nova adicionada na assinatura é chamada \textbf{função de Skolem}, e uma constante nova é chamada \textbf{constante de Skolem}. A skolemização não preserva a equivalência das fórmulas resultante e original, mas preserva a satisfatibilidade: a original é satisfatível se, e somente se, a resultante é satisfatível.

Para ilustrar o processo de remoção de quantificadores, vejamos as seguintes fórmulas:
\begin{itemize}
    \item $\forall xP(x) \rightarrow \exists x S(x)$: \\
    $\equiv \neg \forall xP(x) \lor \exists x S(x)$ \\
    $\equiv \exists x (\neg P(x)) \lor \exists xS(x)$ \\
    $\equiv \underline{\exists x}(\neg P(x) \lor S(x))$ \littlelabel{FNP} \\
    $\exists x \Rightarrow$ adicionando uma constante de Skolem $a$ na assinatura; \\
    $\equiv \neg P(a) \lor S(a)$ \littlelabel{FPS}
    \item $\exists x(R(x,y) \rightarrow \forall y(P(z,y) \land \exists w(S(w,u) \lor \neg R(w,y))))$: \\
    $\equiv \exists x(\neg R(x,y) \lor \forall y(P(z,y) \land \exists w(S(w,u) \lor \neg R(w,y))))$ \\
    $\equiv \exists x(\neg R(x,y) \lor \forall m(P(z,m) \land \exists w(S(w,u) \lor \neg R(w,m))))$ \\
    $\equiv \exists x\forall m(\neg R(x,y) \lor (P(z,m) \land \exists w(S(w,u) \lor \neg R(w,m))))$ \\
    $\equiv \exists x\forall m(\neg R(x,y) \lor \exists w(P(z,m) \land (S(w,u) \lor \neg R(w,m))))$ \\
    $\equiv \underline{\exists x}\forall m\underline{\exists w}(\neg R(x,y) \lor (P(z,m) \land (S(w,u) \lor \neg R(w,m))))$ \littlelabel{FNP} \\
    $\exists x \Rightarrow$ adicionando uma constante de Skolem $a$ na assinatura; \\
    $\exists w \Rightarrow$ adicionando uma função unária de Skolem $f(-)$ na assinatura; \\
    $\equiv \forall m(\neg R(a,y) \lor (P(z,m) \land (S(f(m),u) \lor \neg R(f(m),m))))$ \littlelabel{FPS}
\end{itemize}

Assim, para colocar as fórmulas na forma normal conjuntiva, primeiro colocamos-as na forma padrão de Skolem e simplesmente ignoramos os quantificadores universais.
\begin{rexercises} 
    \begin{question}
        Dadas as sentenças a seguir:
        \begin{enumerate}
            \item[1:] Se o unicórnio é lenda, é imortal, mas se não é lenda, é mamífero.
            \item[2:] O unicórnio, se é imortal ou mamífero, é chifrudo.
            \item[3:] O unicórnio, se é chifrudo, é bruxaria.
            \item[4:] O unicórnio é bruxaria.
            \item[5:] O unicórnio é chifrudo.
        \end{enumerate}
        Prove que $\{1, 2, 3\} \vDash 4 \land 5$.
        \begin{resolution}
            Esse é o mesmo conjunto de sentenças visto no início da 2ª parte, e como vimos, estas são as sentenças na lógica de primeira ordem:
            \begin{enumerate}
                \item[1:] $(L(u) \rightarrow I(u)) \land (\neg L(u) \rightarrow M(u))$ 
                \item[2:] $(I(u) \lor M(u)) \rightarrow C(u)$
                \item[3:] $C(u) \rightarrow B(u)$
                \item[4:] $B(u)$
                \item[5:] $C(u)$
            \end{enumerate} 
        
            Queremos provar que $\{1, 2, 3\} \vDash 4 \land 5$. Sabemos que Resolução não responde se uma fórmula é consequência lógica de um conjunto de premissas, então utilizaremos a equivalência:
            \begin{center}
                $\Gamma \vDash \varphi$ se, e somente se, $\Gamma \cup \{\neg \varphi\}$ é insatisfatível
            \end{center}
            Assim, queremos responder se $\{1, 2, 3, \neg (4 \land 5)\}$ é insatisfatível. Vamos transformar cada sentença do conjunto para a sua forma normal conjuntiva:

            \begin{itemize}
                \item $(L(u) \rightarrow I(u)) \land (\neg L(u) \rightarrow M(u))$
                \\ $\equiv (\neg L(u) \lor I(u)) \land (L(u) \lor M(u))$ \littlelabel{FNC}
                \item $(I(u) \lor M(u)) \rightarrow C(u)$
                \\ $\equiv \neg(I(u) \lor M(u)) \lor C(u)$
                \\ $\equiv (\neg I(u) \land \neg M(u)) \lor C(u)$
                \\ $\equiv (\neg I(u) \lor C(u)) \land (\neg M(u) \lor C(u))$ \littlelabel{FNC}
                \item $C(u) \rightarrow B(u)$
                \\ $\equiv \neg C(u) \lor B(u)$ \littlelabel{FNC}
                \item $\neg (B(u) \land C(u))$
                \\ $\equiv \neg B(u) \lor \neg C(u)$ \littlelabel{FNC}
            \end{itemize}

            Assim:
            \begin{center}
                $(\neg L(u) \lor I(u))^{C_1} \land (L(u) \lor M(u))^{C_2} \land (\neg I(u) \lor C(u))^{C_3} \land (\neg M(u) \lor C(u))^{C_4} \land (\neg C(u) \lor B(u))^{C_5} \land (\neg B(u) \lor \neg C(u))^{C_6}$
            \end{center}

            \begin{itemize}
                \item[] \propagate{5}{6}{}{\neg C(u)^{C_7}}
                \item[] \propagate{7}{4}{}{\neg M(u)^{C_8}}
                \item[] \propagate{8}{2}{}{L(u)^{C_9}}
                \item[] \propagate{9}{1}{}{I(u)^{C_{10}}}
                \item[] \propagate{10}{3}{}{C(u)^{C_{11}}}
                \item[] \propagate{11}{7}{}{()}  
            \end{itemize}
            Encontrada a cláusula vazia, temos que $\{1, 2, 3, \neg (4 \land 5)\}$ é insatisfatível e, portanto, $\{1, 2, 3\} \vDash 4 \land 5$.
        \end{resolution}
    \end{question}

    \begin{question}
        Prove que se uma relação binária é reflexiva e circular, então ela é simétrica.
        \begin{resolution}
            Iremos tomar o símbolo $R$ para representar a relação. Escrevendo as definições de relação reflexiva, circular e simétrica em lógica de primeira ordem, temos:
            \begin{enumerate}
                \item[1:] $\forall x R(x,x)$
                \item[2:] $\forall x\forall y\forall z((R(x, y) \land R(y, z) \rightarrow R(z, x)))$
                \item[3:] $\forall x\forall y(R(x,y) \rightarrow R(y, x))$
            \end{enumerate} 
            Queremos provar que $\{1, 2\} \vDash 3$. Usaremos a equivalência:
            \begin{center}
                $\Gamma \vDash \varphi$ se, e somente se, $\Gamma \cup \{\neg \varphi\}$ é insatisfatível
            \end{center}
            Ou seja, queremos provar que $\{1, 2, \neg 3\}$ é insatisfatível. Transformando cada sentença para sua forma normal conjuntiva:
            \begin{itemize}
                \item $\forall x R(x,x)$
                \\ $\equiv R(x, x)$ \littlelabel{FNC}
                \item $\forall x\forall y\forall z((R(x, y) \land R(y, z)) \rightarrow R(z, x))$
                \\ $\equiv (R(x, y) \land R(y, z)) \rightarrow R(z, x)$
                \\ $\equiv \neg (R(x, y) \land R(y, z)) \lor R(z, x)$
                \\ $\equiv \neg R(x, y) \lor \neg R(y, z) \lor R(z, x)$  \littlelabel{FNC}
                \item $\neg \forall x\forall y(R(x,y) \rightarrow R(y, x))$
                \\ $\equiv \exists x \exists y \neg(R(x, y) \rightarrow R(y, x))$ \littlelabel{FNP}
                \\ $\exists x \Rightarrow$ adicionando uma constante de Skolem $a$ na assinatura;
                \\ $\exists y \Rightarrow$ adicionando uma constante de Skolem $b$ na assinatura;
                \\ $\equiv \neg(R(a, b) \rightarrow R(b, a))$ \littlelabel{FPS}
                \\ $\equiv \neg(\neg R(a, b) \lor R(b, a))$
                \\ $\equiv R(a, b) \land \neg R(b, a)$ \littlelabel{FNC}
            \end{itemize}
            Assim:
            \[R(x, x)^{C_1} \land (\neg R(x, y) \lor \neg R(y, z) \lor R(z, x))^{C_2} \land  R(a, b)^{C_3} \land \neg R(b, a)^{C_4}\]
            \begin{itemize}
                \item [] \propagate{1}{4}{}{\text{não são unificáveis}}
                \item [] \propagate{2}{3}{, pela unificadora $\Theta = [a\diagup x, b\diagup y]$}{(\neg R(b, z) \lor R(z, a))^{C_5}}
                \item [] \propagate{5}{4}{, pela unificadora $\Theta = [b\diagup z]$}{\neg R(b, b)^{C_6}}
                \item [] \propagate{6}{1}{, pela unificadora $\Theta = [b\diagup x]$}{()}
            \end{itemize}
            Encontrada a cláusula vazia, temos que $\{1, 2, \neg 3\}$ é insatisfatível e, portanto, $\{1, 2\} \vDash 3$.
        \end{resolution}
    \end{question}
\end{rexercises}

\begin{exercises}
    \begin{question}
       Para cada uma das fórmulas abaixo, transforme-a para a sua forma normal prenex e, em seguida, aplique a skolemização na fórmula resultante para obter a sua forma padrão de Skolem.
       
       \begin{enumerate}
           \item $\exists x(B(x) \land \forall y K(x, y)) \rightarrow \forall z(T(z) \land \neg \forall x B(x))$
           \item $\forall x \exists y(\exists w (P(w) \land \neg Q(y)) \rightarrow \exists w \forall k (T(k, x) \lor P(w)))$
           \item $\forall y(\forall x P(x, y) \rightarrow \exists z Q(x, z))$
           \item $\exists x \forall y P(x, y) \lor \neg \exists y(Q(y) \rightarrow \forall z R(z))$
       \end{enumerate}
    \end{question}

    \begin{question}
        Construa uma estrutura que sirva como contraexemplo para mostrar que $\exists x P(x) \land \exists x Q(x) \not\equiv \exists x(P(x) \land Q(x))$.
    \end{question}

    \begin{question}
        Construa uma estrutura que sirva como contraexemplo para mostrar que $\forall x P(x) \lor \forall x Q(x) \not\equiv \forall x(P(x) \lor Q(x))$.
    \end{question}

    \begin{question}
        Use o algoritmo de Herbrand para determinar se os conjuntos de termos abaixo são unificáveis. Em caso positivo, dê a substituição unificadora mais geral.
        \begin{enumerate}
            \item $\{h(f(g(y))), h(f(g(g(a)))), h(y)\}$
            \item $\{q(z,x,f(y)), q(z, h(z,w), f(w)), q(z, h(a, g(b)), f(g(b)))\}$
            \item $\{h(f(x, y), g(f(a, y))), h(f(a,y), g(z)), h(f(a,b), g(w))\}$
            \item $\{f(g(x, h(z))), f(g(x, h(x))), f(g(a, h(a)))\}$
        \end{enumerate}
    \end{question}

    \begin{question}
        Execute o Método da Resolução para responder às instâncias do Problema da Satisfatibilidade a seguir.
        \begin{enumerate}
            \item $\{P(x, a) \lor Q(x), \neg P(f(y), a) \lor Q(y), \neg Q(f(b))\} \vDash Q(b)$?
            \item $\{\forall x(P(x) \rightarrow \exists yQ(x, y)), \forall x\forall y \forall z(Q(x, z) \rightarrow P(f(y)))\} \vDash \forall x(P(x) \rightarrow \exists yP(f(y)))$?
            \item $\{\forall x(\exists y(S(x, y) \land M(y)) \rightarrow \exists y(I(y) \land E(x,y)))\} \vDash \neg\exists xI(x) \rightarrow \forall x\forall y(S(x, y) \rightarrow \neg M(y))$?
        \end{enumerate}
    \end{question}

    \begin{question}
        $\star$ Crie uma assinatura e uma estrutura apropriadas para representar as sentenças a seguir, e use o Método da Resolução para mostrar que $\{1, 2, 3, 4, 5\} \vDash 6$, onde:
        \begin{enumerate}
            \item [1:] Todo cachorro uiva à noite.
            \item [2:] Qualquer um que tenha gato em casa não tem nenhum rato em casa.
            \item [3:] Todos que têm sono leve, não têm animal de estimação que uiva à noite em casa.
            \item [4:] João tem um animal de estimação em casa.
            \item [5:] Todo animal de estimação é um gato ou um cachorro.
            \item [6:] Se João tem sono leve, então João não tem ratos em casa.
        \end{enumerate}
    \end{question}
\end{exercises}