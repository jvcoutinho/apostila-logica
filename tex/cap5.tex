\chapter{Sintaxe}

Vimos que, além dos conectivos e parênteses, a lógica de predicados envolve mais quatro tipos de símbolos:

\begin{description}
    \item[Constantes e variáveis] São funções de aridade zero e referenciam um objeto do domínio.
    \item[Predicados e relações] Usamos esses símbolos para denotar alguma propriedade de objetos (predicados) ou uma relação entre objetos (relações). Gottlob Frege, o fundador da lógica de predicados, mostrou como esses conceitos podem ser representados por funções proposicionais (ou seja, funções que retornam verdadeiro ou falso) mesmo que, na prática, sejam conjuntos.
    \item[Funções] Símbolos que representam funções de referência indireta, ou seja, servem para referenciar um objeto a partir de outros.      
    \item[Quantificadores] Símbolos que denotam quantidade: $\exists$ para denotar ``algum objeto'' e $\forall$ para denotar ``todo objeto''.
\end{description}
Assim, o alfabeto $\Sigma$ da lógica de primeira ordem consiste na união desses quatro tipos de símbolos, acrescidos dos conectivos lógicos e parênteses.

\section{Fórmulas}

Assim como as expressões legítimas da lógica proposicional são chamadas proposições, as expressões legítimas da lógica de primeira ordem são chamadas \textbf{fórmulas}.

A unidade básica de uma fórmula é a \textbf{fórmula atômica}, que é uma relação entre objetos. Para representar objetos, tomamos o conceito de \textbf{termos}.
\begin{definition}{Termos}
    Seja $L$ uma linguagem. O conjunto de termos de $L$ é definido indutivamente da seguinte forma:
    \begin{itemize}
        \item Todo símbolo de constante $c$ de $L$ é um termo;
        \item Toda variável é um termo;
        \item Se $f$ for um símbolo de função $n$-ária de $L$ e se $t_1,...,t_n$ forem termos de $L$, então $f(t_1,...,t_n)$ é um termo. 
    \end{itemize}
    Um termo que não contém variáveis é dito \textbf{termo fechado}.
    \tcbsubtitle{Fórmulas Atômicas}
    Seja $L$ uma linguagem. Uma fórmula atômica é uma palavra sobre o vocabulário simbólico de $L$ com um dos dois formatos:
    \begin{itemize}
        \item $R(t_1,...t_n)$, onde $R$ é um símbolo de relação $n$-ária de $L$ e $t_1,...,t_n$ são termos de $L$;
        \item $t_1 = t_2$, onde $t_1$ e $t_2$ são termos de $L$.
    \end{itemize} 
    Uma fórmula atômica que não contém variáveis é dita \textbf{sentença atômica}.
\end{definition}

E assim, podemos definir indutivamente o conjunto das expressões legítimas da lógica de predicados, chamado de conjunto das \textbf{fórmulas bem formadas} ($FORM$).
\begin{definition}{Fórmula bem formada}
    \begin{itemize}
        \item Toda fórmula atômica é uma fórmula bem formada;
        \item Se $\omega$ é uma fórmula bem formada, então $(\neg \omega)$ é uma fórmula bem formada;
        \item Se $\omega_1$ e $\omega_2$ são fórmulas bem formadas, então $(\omega_1 \land \omega_2)$ é uma fórmula bem formada.
        \item Se $\omega_1$ e $\omega_2$ são fórmulas bem formadas, então $(\omega_1 \lor \omega_2)$ é uma fórmula bem formada.
        \item Se $\omega_1$ e $\omega_2$ são fórmulas bem formadas, então $(\omega_1 \rightarrow \omega_2)$ é uma fórmula bem formada.
        \item Se $\omega$ é uma fórmula bem formada e $x$ é uma variável livre em $\omega$, então $(\forall x\omega)$ é uma fórmula bem formada;
        \item Se $\omega$ é uma fórmula bem formada e $x$ é uma variável livre em $\omega$, então $(\exists x\omega)$ é uma fórmula bem formada.
    \end{itemize}

    Uma fórmula que não contém variáveis livres é dita \textbf{sentença}.
\end{definition}

Veremos a definição de variável livre a seguir. Temos então que $FORM$ é o fecho indutivo do conjunto base $X$ de fórmulas atômicas sob o conjunto de funções geradoras $F = \{f_\forall, f_\exists, f_\neg, f_\lor, f_\land, f_\rightarrow\}$, e é possível mostrar que, da forma como definimos, $FORM$ é livremente gerado.
\section{Variáveis}

Uma variável, como vimos, é a representação de um objeto do domínio. Dada uma fórmula $(Q x\omega)$, onde $Q$ é $\forall$ ou $\exists$, dizemos que o \textbf{escopo} do quantificador $Q x$ é $\omega$. Por exemplo, na fórmula a seguir:
\[S(x) \lor \exists z(P(z) \land \forall x(R(x,y) \rightarrow \exists yR(y,x)))\]
O escopo de $\exists z$ é $(P(z) \land \forall x(R(x,y) \rightarrow \exists yR(y,x))$, o de $\forall x$ é $(R(x,y) \rightarrow \exists yR(y,x))$ e o de $\exists y$ é apenas $R(y,x)$. 
Uma ocorrência de uma variável em uma fórmula é dita \textbf{ligada} se, e somente se, a variável está dentro do escopo de um quantificador aplicado a ela ou ela é a ocorrência do quantificador. Uma ocorrência de uma variável em uma fórmula é dita \textbf{livre} se, e somente se, essa ocorrência não é ligada. 
\begin{center}
    $S({\color{Green}x}) \lor \exists {\color{Peach}z}(P({\color{Peach}z}) \land \forall {\color{Peach}x}(R({\color{Peach}x},{\color{Green}y}) \rightarrow \exists {\color{Peach}y}R({\color{Peach}y},{\color{Peach}x})))$
\end{center}

Assim, dizemos que uma variável é ligada em uma fórmula se há pelo menos uma ocorrência ligada dela na fórmula e, similarmente, uma variável é livre em uma fórmula se há pelo menos uma ocorrência livre dela na fórmula. Na fórmula anterior, $x$ e $y$ são variáveis {\color{Green}livres}, enquanto $x$, $y$ e $z$ são variáveis {\color{Peach}ligadas} (é possível que uma variável seja livre e ligada ao mesmo tempo em uma fórmula, mas em ocorrências diferentes).

Podemos definir uma função recursiva que obtém o conjunto de variáveis livres em uma fórmula:
\begin{definition}{Conjunto das variáveis livres em uma fórmula}
    $VL: FORM \mapsto \mathcal{P}(\text{VARIÁVEIS})$ \\
    $VL(\varphi) = \{x_1,...,x_n\}$, se $\varphi$ é atômica e $x_1,...,x_n$ ocorrem em $\varphi$ \\
    $VL((\neg \psi)) = VL(\psi)$ \\
    $VL((\rho \land \theta)) = VL(\rho) \cup VL(\theta)$ \\
    $VL((\rho \lor \theta)) = VL(\rho) \cup VL(\theta)$ \\
    $VL((\rho \rightarrow \theta)) = VL(\rho) \cup VL(\theta)$ \\
    $VL((\forall x\omega)) = VL(\omega) - \{x\}$ \\
    $VL((\exists x\omega)) = VL(\omega) - \{x\}$
\end{definition}

\subsection{Substituição de variáveis}

Para atribuir um valor $t$ à uma variável livre $x$ em uma fórmula $\varphi$ (notação $\varphi[t\diagup x]$), devemos substituir todas as ocorrências livres dessa variável na fórmula por esse valor. Assim, queremos definir precisamente esse processo. Temos duas funções: uma aplicada a termos e uma aplicada a fórmulas.
\begin{definition}{Substituição de uma variável $x$ por um termo $t$ em um termo $s$}
    $s[t\diagup x]$$: \text{TERM} \times \text{TERM} \times \text{VARIÁVEIS} \mapsto \text{TERM}$ \\
    $x[t\diagup x] = t$ \\
    $y[t\diagup x] = y \text{, se } x \neq y$ \\
    $c[t\diagup x] = c$ \\
    $f(t_1,...,t_n)[t\diagup x] = f(t_1[t\diagup x],...,t_n[t\diagup x])$
    \tcbsubtitle{Substituição de uma variável $x$ por um termo $t$ em uma fórmula $\varphi$} 
    $\varphi[t\diagup x]$$: \text{FORM} \times \text{TERM} \times \text{VARIÁVEIS} \mapsto \text{FORM}$ \\
    $R(t_1,...,t_n)[t\diagup x] = R(t_1[t\diagup x],...,t_n[t\diagup x])$ \\
    $(t_1 = t_2)[t\diagup x] = (t_1[t\diagup x] = t_2[t\diagup x])$ \\
    $(\neg \psi)[t\diagup x] = (\neg \psi[t\diagup x])$ \\
    $(\rho \land \theta)[t\diagup x] = (\rho[t\diagup x] \land \theta[t\diagup x])$ \\
    $(\rho \lor \theta)[t\diagup x] = (\rho[t\diagup x] \lor \theta[t\diagup x])$ \\
    $(\rho \rightarrow \theta)[t\diagup x] = (\rho[t\diagup x] \rightarrow \theta[t\diagup x])$ \\
    $(\forall x \omega)[t\diagup x] = (\forall x \omega)$ \\
    $(\forall y \omega)[t\diagup x] = (\forall y \omega[t\diagup x])$, se $x \neq y$ \\
    $(\exists x \omega)[t\diagup x] = (\exists x \omega)$ \\
    $(\exists y \omega)[t\diagup x] = (\exists y \omega[t\diagup x])$, se $x \neq y$
\end{definition}

A função de substituição explora recursivamente a fórmula até encontrar uma ocorrência livre da variável a ser substituída, e então a substitui -- isso implica que ocorrências ligadas da variável são ignoradas.

Porém, há uma condição especial para substituição que devemos considerar. Observe a seguinte fórmula $\varphi$:
\[\forall x(x = y)\]
Suponha que $A$ seja uma estrutura com mais de um elemento em seu domínio. Isso significa que, para qualquer valor $a$ que atribuirmos a $y$, a fórmula não diz a verdade nessa estrutura pois nem todo elemento de $A$ é $a^A$. Isso implica (veremos com mais detalhes no próximo capítulo), que a fórmula $\varphi$ não é \textbf{válida}. Porém, ao substituirmos $y$ pela variável $x$, temos:
\[\forall x(x = x)\]
Nesse caso, a fórmula sempre diz a verdade, independemente da estrutura analisada, pois qualquer elemento em qualquer estrutura é igual a ele mesmo. Desse modo, a substituição causou uma fórmula não válida se tornar válida, e isso não deve acontecer. Assim, não devemos realizar substituições de um termo $t$ em uma variável $x$ que \textbf{causariam ocorrências livres de uma variável em $t$ se tornarem ligadas após a substituição}. Temos então, a noção de \textbf{termo livre}.
\begin{definition}{Termo livre}
    Dizemos que um termo $t$ está livre para entrar no lugar da variável $x$ em uma fórmula $\varphi$ se:
    \begin{itemize}
        \item $\varphi$ é atômica;
        \item $\varphi$ é da forma $(\neg \psi)$ e $t$ está livre para entrar no lugar de $x$ em $\psi$;
        \item $\varphi$ é da forma $(\rho * \theta)$, $t$ está livre para entrar no lugar de $x$ em $\rho$ e em $\theta$ e $* \in \{\land, \lor, \rightarrow\}$;
        \item $\varphi$ é da forma $(\forall y\omega)$ ou $(\exists y\omega)$, $x = y$ ou $x \neq y$ e $x \notin VL(t)$, e $t$ está livre para entrar no lugar de $x$ em $\omega$.
    \end{itemize}
\end{definition}
Assim, executamos essa função antes de proceder com a função de substituição. Na fórmula $\forall x(x = y)$, não podemos realizar a substituição $[x\diagup y]$ pois $x \neq y$, mas $x \in VL(x)$.
\begin{rexercises}
    \begin{question}
        O termo $f(y,z)$ está livre para entrar no lugar da variável $y$ na fórmula\\ $\forall x(R(y, x) \rightarrow \forall z S(z, y))$? E na fórmula $\forall x(R(y, x) \rightarrow \forall y S(z, y))$?
        \begin{resolution}
            Observe que o enunciado é equivalente a perguntar se as substituições $[f(y,z)\diagup y]$ nas fórmulas $\forall x(R(y, x) \rightarrow \forall z S(z, y))$ e $\forall x(R(y, x) \rightarrow \forall y S(z, y))$ são possíveis de serem realizadas seguramente. 
            
            Para a primeira fórmula, vemos rapidamente que não, pois a variável $z$ em $f(y,z)$ se tornaria ligada, o que não deve acontecer. Para a segunda fórmula, a única ocorrência da variável $y$ que será substituída é a em $R(y,x)$, pois a em $S(z,y)$ é uma ocorrência ligada. Desse modo, as variáveis de $f(y,z)$ não se tornariam ligadas e, portanto, a substituição é possível.
            Podemos também responder ambas as questões aplicando a função recursiva.
        \end{resolution}
    \end{question}
\end{rexercises}

\begin{exercises}
    \begin{question}
        Calcule o conjunto de variáveis livres da fórmula a seguir usando a função recursiva definida no capítulo.
        \[\forall x(P(x, y) \rightarrow \exists y(R(z,  y) \lor \exists w(S(w, u) \land \neg R(w, y))))\]
    \end{question}
    \begin{question}
        Repita a questão anterior, mas dessa vez calcule o conjunto de variáveis ligadas. Defina uma função recursiva para tal e aplique-a na fórmula.
    \end{question}
    \begin{question}
        Determine se as substituições a seguir são possíveis de serem realizadas. Se sim, aplique.
        \begin{enumerate}
            \item $\forall x(R(y, x) \rightarrow S(y))$ $[f(y, z) \diagup y]$
            \item $\forall z\forall x(R(y, x) \rightarrow S(y))$ $[f(y, z) \diagup y]$
            \item $\forall x(P(x) \lor Q(x, f(y)))$ $[g(y) \diagup y]$
            \item $\forall x(P(x) \lor Q(x, f(y)))$ $[g(x) \diagup y]$
            \item $\forall x(R(x) \land S(z))$ $[g(z) \diagup y]$
            \item $\forall x(R(x) \land S(z))$ $[g(z)\diagup z]$
        \end{enumerate}
    \end{question}
    \begin{question}
        Determine se as afirmações a seguir são verdadeiras ou falsas.
        \begin{enumerate}
            \item Se $R$ é um símbolo de relação unária, $f$ é um de função binária e $a$ e $b$ são de constantes, então $R(f(a,b))$ é um termo fechado.
            \item Se $f$ é um símbolo de função unária e $x$ e $y$ são variáveis, então $f(x) = f(y)$ é uma fórmula atômica.
            \item Se $f$ é um símbolo de função unária e $x$ e $y$ são variáveis, então $f(x) = f(y)$ é uma sentença atômica.
            \item Se $f$ é um símbolo de função $n$-ária e $x_1,...,x_n$ são variáveis, então $f(x_1,...,x_n)$ é uma fórmula atômica.
        \end{enumerate}
    \end{question}
\end{exercises}