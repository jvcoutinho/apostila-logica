\chapter{Sintaxe}

Vimos que, além dos conectivos e parênteses, a lógica de predicados envolve mais quatro tipos de símbolos:

\begin{description}
    \item[Constantes e variáveis] São funções de aridade zero e referenciam um objeto do domínio.
    \item[Predicados e relações] Usamos esses símbolos para denotar alguma propriedade de objetos (predicados) ou uma relação entre objetos (relações). Gottlob Frege, o fundador da lógica de predicados, mostrou como esses conceitos podem ser representados por funções proposicionais (ou seja, funções que retornam verdadeiro ou falso) mesmo que, na prática, sejam conjuntos.
    \item[Funções] Símbolos que representam funções de referência indireta, ou seja, servem para referenciar um objeto a partir de outros.      
    \item[Quantificadores] Símbolos que denotam quantidade: $\exists$ para denotar ``algum objeto'' e $\forall$ para denotar ``todo objeto''.
\end{description}
Assim, o alfabeto $\Sigma$ da lógica de primeira ordem consiste na união desses quatro tipos de símbolos, acrescidos dos conectivos lógicos e parênteses.

\section{Fórmulas}

Assim como as expressões legítimas da lógica proposicional são chamadas proposições, as expressões legítimas da lógica de primeira ordem são chamadas \textbf{fórmulas}.

A unidade básica de uma fórmula é a \textbf{fórmula atômica}, que é uma relação entre objetos. Para representar objetos, tomamos o conceito de \textbf{termos}.
\begin{definition}{Termos}
    Seja $L$ uma linguagem. O conjunto de termos de $L$ é definido indutivamente da seguinte forma:
    \begin{itemize}
        \item Todo símbolo de constante $c$ de $L$ é um termo;
        \item Toda variável é um termo;
        \item Se $f$ for um símbolo de função $n$-ária de $L$ e se $t_1,...,t_n$ forem termos de $L$, então $f(t_1,...,t_n)$ é um termo. 
    \end{itemize}
    Um termo que não contém variáveis é dito \textbf{termo fechado}.
    \tcbsubtitle{Fórmulas Atômicas}
    Seja $L$ uma linguagem. Uma fórmula atômica é uma palavra sobre o vocabulário simbólico de $L$ com um dos dois formatos:
    \begin{itemize}
        \item $R(t_1,...t_n)$, onde $R$ é um símbolo de relação $n$-ária de $L$ e $t_1,...,t_n$ são termos de $L$;
        \item $t_1 = t_2$, onde $t_1$ e $t_2$ são termos de $L$.
    \end{itemize} 
    Uma fórmula atômica que não contém variáveis é dita \textbf{sentença atômica}.
\end{definition}