\chapter{Estruturas}

Como vimos, o vocabulário simbólico da lógica de predicados inclui símbolos para representar objetos e predicados, além de símbolos para os conectivos. Assim, a noção de valoração-verdade é incompatível com a lógica de primeira ordem, e necessitamos enriquecê-la para algo que nos permita atribuir valores aos objetos e predicados. Tomamos então, o conceito de \textbf{estrutura matemática}.
\begin{definition}{Estrutura Matemática}
    Uma estrutura $A$ é definida por 4 componentes:
    \begin{itemize}
        \item Conjunto de objetos chamado de \textbf{domínio} ou \textbf{universo} de $A$ -- denotado por $dom(A)$.
        \item Subconjunto de elementos de $dom(A)$ considerados \textbf{destaques} ou \textbf{constantes}.
        \item Conjunto de \textbf{relações} sobre $dom(A)$, cada uma com sua aridade.
        \item Conjunto de \textbf{funções} sobre $dom(A)$, cada uma com sua aridade.
    \end{itemize}
\end{definition}

Para ilustrar o conceito, eis três exemplos:
\begin{description}
    \item[A:]
     
    \begin{center}
        \begin{structure}
            {$\mathbb{N}$}
            {}
            {$<$$(-,-)$}
            {$0$\\$1$}
            {sucessor$(-)$\\soma$(-,-)$}
        \end{structure}
    \end{center} 

    \item[B:]
        \begin{center}
            \begin{structure}
                {Estados do Brasil}
                {}
                {tem-litoral$(-)$\\mais-rico-que$(-,-)$}
                {SP\\PE}
                {mais-populoso$(-,-)$}
            \end{structure}
        \end{center} 

        \item[C:]
        \begin{center}
            \begin{structure}
                {$\mathbb{N}$}
                {}
                {divide$(-,-)$\\primo$(-)$}
                {2\\3}
                {quadrado$(-)$}
            \end{structure}
        \end{center} 
\end{description}

Uma vez definida uma estrutura, podemos criar um vocabulário simbólico sobre a estrutura que nos permita codificar sentenças na lógica de predicados. Tal vocabulário deve dizer quão rica ou simples é a estrutura, envolvendo o número de relações, destaques e funções. Chamamos esse vocabulário de \textbf{assinatura} da estrutura.
\begin{definition}{Assinatura}
    Seja $A$ uma estrutura. A assinatura $L$ de $A$ é definida pelos seguintes componentes:
    \begin{itemize}
        \item Quantidade de símbolos de destaques e os símbolos.
        \item Quantidade de símbolos de relações $n$-árias, onde $n \in \mathbb{N}$, e os símbolos.
        \item Quantidade de símbolos de funções $n$-árias, onde $n \in \mathbb{N}$, e os símbolos. 
    \end{itemize}
    $A$ é dita $L$-Estrutura.
\end{definition} 

A assinatura diz respeito somente à quantidade de símbolos. A definição dos mesmos é feita na \textbf{linguagem}. Mas, por simplicidade, unimos os dois conceitos. Assim, podemos definir a assinatura de $A$ como:
\begin{itemize}
    \item 2 símbolos de destaques: $a$ e $b$;
    \item 1 símbolo de relação binária: $R$;
    \item 1 símbolo de função unária: $f$;
    \item 1 símbolo de função binária: $g$.
\end{itemize}
Uma vez definidos os símbolos, precisamos dizer o que eles representam em uma estrutura, para que possamos avaliar as sentenças que usam esses símbolos. Tal processo chama-se \textbf{interpretação}.
\begin{definition}{Interpretação}
    Seja $L$ uma assinatura e $A$ uma $L$-Estrutura. A interpretação de $L$ em $A$ é uma associação de cada símbolo de $L$ a um elemento de cada componente de $A$, tal que:

    \begin{itemize}
        \item A cada símbolo $c$ de constante, associa-se um elemento destacado do domínio de $A$ (notação $c^A$).
        \item A cada símbolo $R$ de relação de aridade $n$, associa-se uma relação de $A$ de aridade $n$ (notação $R^A$).
        \item A cada símbolo $f$ de relação de aridade $n$, associa-se uma função de $A$ de aridade $n$ (notação $f^A$).
    \end{itemize}
\end{definition}

Assim, podemos tomar a seguinte interpretação da assinatura de $A$ em $A$:
\begin{itemize}
    \item $a^A = 0$ e $b^A = 1$;
    \item $R^A =$ $<$$(-,-)$;
    \item $f^A =$ sucessor$(-)$ e $g^A =$ soma$(-,-)$.
\end{itemize}
Podemos então formalizar sentenças sobre a estrutura $A$ na lógica de predicados:
\begin{description}
        \item[$2$ é menor que $3$.] $R(f(b), f(f(b)))$ 
        \item[Para todo natural $x$, há um natural $y$ maior que ele.] $\forall x\exists y(R(x,y))$
        \item[Para todo natural $x$, a soma entre $1$ e $x$ é igual ao sucessor de $x$.] $\forall x(g(b, x) = f(x))$ 
        \item[0 não é sucessor de nenhum natural.] $\neg\exists x(f(x) = a)$    
        \item[Para todo natural $x$, existem dois naturais cuja soma é $x$.] $\forall x\exists y\exists z(g(y,z) = x)$  
\end{description}
\section{Subestruturas}

Como saber se uma estrutura $A$ é subestrutura de uma estrutura $B$? Se $A$ e $B$ forem simplesmente conjuntos, basta saber se todos os elementos de $A$ também são elementos de $B$. Mas, considerando os outros componentes das estruturas $A$ e $B$ (relações, destaques e funções), é necessário verificar se esses componentes possuem uma relação entre si que justifique dizer que $A$ está contida em $B$ como estrutura.

Para definir matematicamente esse possível relacionamento, tomamos emprestado da álgebra a noção de \textbf{homomorfismo}: uma função que preserva propriedades.

\begin{definition}{Homomorfismo}
    Sejam $A$ e $B$ estruturas de uma mesma assinatura $L$. Uma função $h: dom(A) \mapsto dom(B)$ é dita homomorfismo de $A$ para $B$ se as condições seguintes forem satisfeitas.
    \begin{enumerate}
        \item Para todo símbolo de constante $c$ de $L$, $h(c^A) = c^B$;
        \item Para todo símbolo de relação $n$-ária $R$ de $L$ e toda $n$-upla $(a_1,...,a_n)$ de elementos de $A$, $(a_1,...a_n) \in R^A \rightarrow (h(a_1),...,h(a_n)) \in R^B$;
        \item Para todo símbolo de função $n$-ária $f$ de $L$ e toda $n$-upla $(a_1,...,a_n)$ de elementos de $A$, $h(f^A(a_1,...a_n)) = f^B(h(a_1),...,h(a_n))$.  
    \end{enumerate}
\end{definition}

Para ilustrar esse conceito, tomemos duas estruturas $A$ e $B$:

\begin{table}[h]
    \centering
    \begin{tabular}{c c}
        $\mathbf{A}$ & $\mathbf{B}$ \\
        \begin{structure}
            {}
            {$0$ $1$ $3$ $5$}
            {$R_1(-,-)$\\$R_2(-)$}
            {$1$ $3$}
            {$f(-)$}
        \end{structure}
        &
        \begin{structure}
            {}
            {$0$ $1$ $2$ $3$ $5$}
            {$R_3(-,-)$\\$R_4(-)$}
            {$1$ $2$}
            {$g(-)$}
        \end{structure} 
    \end{tabular}
\end{table}

Suponha que:
\begin{description}
    \item $R_1 = \{(0,3), (1,3), (3,5), (5, 3)\}$ \quad|\quad $R_3 = \{(0,3), (1,2), (3,5), (2, 3), (3,2), (3,3)\}$
    \item $R_2 = \{0,1,5\}$ \quad|\quad $R_4 = \{0,1,2,3,5\}$
    \item $f(0) = 1, f(1) = 1, f(3) = 2, f(5) = 3$ \quad|\quad $g(0) = 0, g(1) = 1, g(2) = 2, g(3) = 3, g(5) = 5$
\end{description}
Seja $h: dom(A) \mapsto dom(B)$ uma função entre as duas estruturas, definida da seguinte forma:
\begin{center}
    $h(0) = 1$ \\
    $h(1) = 1$ \\
    $h(3) = 2$ \\
    $h(5) = 3$
\end{center}
$h$ é um homomorfismo de $A$ para $B$? Vamos verificar cada condição:
\begin{enumerate}
    \item A 1ª condição diz que os destaques de $A$ são mapeados para destaques de $B$. Notamos que $h(1) = 1$ e $h(3) = 2$. Assim, a 1ª condição é satisfeita e dizemos que $h$ \textbf{preserva destaques}.
    \item A 2ª condição diz que se uma tupla de elementos se relaciona em $A$, então a tupla contendo os mapeamentos desses elementos se relaciona em $B$. Analisando as relações:
    \begin{description}
        \item[$R_1$:] $(0,3) \mapsto (h(0),h(3)) = (1,2) \in R_3$\\
        $(1,3) \mapsto (h(1),h(3)) = (1,2) \in R_3$ \\
        $(3,5) \mapsto (h(3),h(5)) = (2,3) \in R_3$ \\
        $(5,3) \mapsto (h(5),h(3)) = (3,2) \in R_3$
        \item[$R_2$:] $0 \mapsto h(0) = 1 \in R_4$ \\
        $1 \mapsto h(1) = 1 \in R_4$ \\
        $5 \mapsto h(5) = 3 \in R_4$
    \end{description}
    Assim, a 2ª condição é satisfeita e dizemos que $h$ \textbf{preserva relações}.
    \item A 3ª condição diz que mapear a aplicação de uma função em $A$ corresponde a mapear primeiro os argumentos e depois aplicar uma função em $B$. Analisando as funções:
    \begin{center}
        $h(f(0)) = g(h(0)) = g(1) = 1$ \\
        $h(f(1)) = g(h(1)) = g(1) = 1$ \\
        $h(f(3)) = g(h(3)) = g(2) = 2$ \\
        $h(f(5)) = g(h(5)) = g(3) = 3$
    \end{center}
    Assim, a 3ª condição é satisfeita e dizemos que $h$ \textbf{preserva funções}. Por preservar destaques, relações e funções, $h$ é um homomorfismo de $A$ para $B$.
\end{enumerate}
     
\subsection{Imersão}
Um homomorfismo $h: dom(A) \rightarrow dom(B)$ é dito \textbf{imersão} se:
\begin{itemize}
    \item $h$ é injetora;
    \item $h$ satisfaz uma versão mais forte da 2ª condição:
    \\ Para todo símbolo de relação $n$-ária $R$ de $L$ e toda $n$-upla $(a_1,...,a_n)$ de elementos de $A$, $(a_1,...a_n) \in R^A \leftrightarrow (h(a_1),...,h(a_n)) \in R^B$.
\end{itemize}
A função $h$ do exemplo anterior não é uma imersão, uma vez que, não só ela quebra a primeira condição (pois $h(0) = h(1) = 1$, implicando que $h$ não é injetora) como a segunda ($(5,5) \notin R_1$, mas $(h(5),h(5)) = (3,3) \in R_2$).
Além da imersão, existem outras variantes do homomorfismo:
\begin{itemize}
    \item Uma imersão sobrejetora é dita \textbf{isomorfismo}.
    \item Um homomorfismo $h: dom(A) \mapsto dom(A)$ é dito \textbf{endomorfismo} de $A$.
    \item Um isomorfismo $h: dom(A) \mapsto dom(A)$ é dito \textbf{automorfismo} de $A$.
\end{itemize}

Agora, podemos remeter ao problema inicial e definir então as condições para que uma estrutura $A$ seja subestrutura de uma estrutura $B$.
\begin{definition}{Subestrutura}
    Sejam $A$ e $B$ estruturas de mesma assinatura. Dizemos que $A$ é subestrutura de $B$ se:
    \begin{enumerate}
        \item $dom(A) \subseteq dom(B)$
        \item A função identidade $i: dom(A) \mapsto dom(B)$ $|$ $i(x) = x$ é uma imersão.
    \end{enumerate}
    A notação é $A \subseteq B$.
\end{definition}

\subsection{O Problema da Menor Subestrutura}

Seja $A$ a estrutura a seguir e $X = \{0,1,3\}$ um subconjunto do domínio de $A$:
\begin{center}
    \begin{structure}
        {}
        {$0$ $1$ $2$ $3$ $4$ $5$}
        {primo$(-)$\\$<$$(-,-)$}
        {$1$ $2$}
        {quadrado-mod-5$(-)$\\soma-mod-5$(-,-)$}
    \end{structure} 
\end{center}

Queremos construir uma subestrutura de $A$ que contenha o menor número de elementos em seu domínio e que contenha $X$. Estamos diante de um problema de otimização:

\begin{description}
    \item[Dada:] uma $L$-Estrutura $A$ e um conjunto $X \subseteq dom(A)$;
    \item[Pergunta-se:] qual a menor subestrutura $B$ de $A$ que contém $X$, ou seja, $B \subseteq A$ e $X \subseteq dom(B)$?   
\end{description}
A notação que usamos para $B$ é $\langle X \rangle_A$. Assim, $B$ deve conter os mesmos destaques, relações e funções que $A$ e deve conter $X$ em seu domínio. Além disso, precisamos adicionar elementos ao domínio de $B$ para que a definição de estrutura se mantenha consistente. 

Inicialmente, temos a seguinte estrutura:
\begin{center}
    \begin{structure}
        {}
        {$0$ $1$ $3$}
        {primo$(-)$\\$<$$(-,-)$}
        {$1$ $2$}
        {quadrado-mod-5$(-)$\\soma-mod-5$(-,-)$}
    \end{structure} 
\end{center}

Note que ela possui o destaque $2$, que não pertence ao domínio. Por definição, o conjunto de destaques é subconjunto do domínio, logo, devemos adicioná-lo a este:
\begin{center}
    \begin{tikzpicture}[modal]
        \node[world] (Dom) [align=left] {$0$ $1$ $2$ $3$};
    \end{tikzpicture} 
\end{center}

Note também que a função quadrado-mod-5 aplicada a $3$ retorna $4$, que não é um elemento do domínio. Por definição, o domínio é fechado sob as funções, assim, devemos adicionar $4$ ao domínio:

\begin{center}
    \begin{tikzpicture}[modal]
        \node[world] (Dom) [align=left] {$0$ $1$ $2$ $3$ $4$};
    \end{tikzpicture} 
\end{center}

Dessa forma, $\langle X \rangle_A$ é a estrutura com domínio $\{0,1,2,3,4\}$ e com os mesmos destaques, relações e funções que $A$. 

Podemos sintetizar o procedimento para construir $\langle X \rangle_A$ da seguinte forma:
\begin{enumerate}
    \item Inicialmente, adicione os destaques, funções e relações de $A$ e faça $dom(B) = X$.
    \item Adicione os destaques de $B$ ao domínio de $B$.
    \item Repita até que nenhum elemento novo seja adicionado:
    \begin{enumerate}
        \item Adicione os conjuntos imagens das funções de $B$ ao domínio de $B$.
    \end{enumerate}  
\end{enumerate}

\subsection{Extensão de uma estrutura}

Quando $\langle \emptyset \rangle_A = A$, ou seja, a menor subestrutura de $A$ construída a partir do conjunto vazio como domínio é a própria $A$, todos os elementos de $A$ são alcançáveis a partir dos destaques e funções de $A$.

Isso não é verdade na estrutura do exemplo anterior, uma vez que $5$ é um elemento inalcançável a partir dos destaques e funções disponíveis e é considerado ``sem nome'' (é impossível representar 5 por meio de símbolos sobre essa estrutura). Dessa forma, podemos extender a estrutura $A$, adicionando ao seu conjunto de destaques os elementos inacessíveis de $A$. A estrutura resultante $A'$ é chamada de \textbf{extensão} de $A$ (e $A$ é dita \textbf{reduto} de $A'$).
    
A extensão da estrutura do exemplo anterior é, portanto:
\begin{center}
    \begin{structure}
        {}
        {$0$ $1$ $2$ $3$ $4$ $5$}
        {primo$(-)$\\$<$$(-,-)$}
        {$1$ $2$ $5$}
        {quadrado-mod-5$(-)$\\soma-mod-5$(-,-)$}
    \end{structure} 
\end{center}
\begin{rexercises}
    \begin{question}
        Escreva o enunciado abaixo na Lógica de Predicados. Crie uma assinatura e uma estrutura apropriadas para representá-lo, e escolha uma interpretação da assinatura na estrutura.
        \begin{center}
            Todo inteiro par maior que 2 pode ser escrito como uma soma de dois números primos. (\textbf{Conjectura de Goldbach})
        \end{center}
        \begin{resolution}
            Queremos construir uma estrutura para a qual somos capazes de expressar a Conjectura de Goldbach na lógica de primeira ordem usando sua assinatura. Vamos analisar a sentença:
            \begin{itemize}
                \item O enunciado fala de inteiros. É razoável admitir, portanto, que o domínio da estrutura é o conjunto dos inteiros.
                \item 2 é um inteiro que aparece ``destacado'' no enunciado. Desse modo, adicionamos-o ao conjunto de destaques.
                \item Aparecem as relações: par; maior que; primo.
                \item Aparece a função soma.
            \end{itemize}
            Temos, então:
            \begin{center}
                \begin{structure}
                    {$\mathbb{Z}$}
                    {}
                    {par$(-)$\\maior$(-,-)$\\primo$(-)$}
                    {$2$}
                    {soma$(-,-)$}
                \end{structure}
            \end{center}
            Podemos definir sua assinatura da seguinte forma:
            \begin{itemize}
                \item 1 símbolo de destaque: $a$
                \item 2 símbolos de relação unária: $R$, $P$
                \item 1 símbolo de relação binária: $M$
                \item 1 símbolo de função binária: $s$
            \end{itemize}
            E então, com base na seguinte interpretação:
            \begin{itemize}
                \item $a^A = 2$
                \item $R^A =$ Par$(-)$, $P^A =$ Primo$(-)$, $M^A =$ Maior$(-,-)$
                \item $s^A =$ soma$(-,-)$
            \end{itemize}
            Podemos escrever a Conjectura de Goldbach na lógica de primeira ordem:
            \[\forall x( (R(x) \land M(x, a)) \rightarrow \exists y \exists z(P(y) \land P(z) \land x = s(y, z)))\]
        \end{resolution}
    \end{question}

    \begin{question}
        Sejam $A$ e $B$ estruturas de mesma assinatura, definidas a seguir.
        \begin{table}[h]
            \centering
            \begin{tabular}{c c}
                A & B \\
                \begin{structure}
                    {$\mathbb{N}$}
                    {}
                    {divide$(-)$}
                    {$0$}
                    {}
                \end{structure}
                &
                \begin{structure}
                    {$\mathbb{Z}$}
                    {}
                    {menor-ou-igual$(-)$}
                    {$1$}
                    {}
                \end{structure}
            \end{tabular}
        \end{table}
        \begin{enumerate}
            \item Defina um homomorfismo de $A$ para $B$.
            \item Determine se $A$ é subestrutura de $B$.
        \end{enumerate}
        \begin{resolution}
            \begin{enumerate}[leftmargin=*]
                \item Queremos definir uma função $h: dom(A) \mapsto dom(B)$ e provar que essa função é um homomorfismo de $A$ para $B$. Temos:
                
                \begin{center}
                    \begin{tabular}{l}
                        $h$$: dom(A) \mapsto dom(B)$ \\
                        $h(x) = x + 1$
                    \end{tabular}
                \end{center}
                \begin{itemize}
                    \item $h$ preserva destaques, pois os destaques de $A$ são mapeados para destaques de $B$: $h(0) = 1$;
                    \item $h$ preserva relações, pois se um par $(x, y) \in dom(A)$ pertence à relação divide, então $(h(x), h(y)) = (x + 1, y + 1) \in dom(B)$ pertence à relação menor-ou-igual.
                    \item $h$ preserva funções trivialmente, pois não há funções na assinatura das estruturas.
                \end{itemize}
                Por preservar destaques, relações e funções, $h$ cumpre as 3 condições de homomorfismo e, portanto, é um.

                \item Queremos verificar se as estruturas satisfazem as duas condições de subestrutura:
                \begin{enumerate}
                    \item[1.] Como $\mathbb{N} \subseteq \mathbb{Z}$, a primeira condição é satisfeita.
                    \item[2.] Vamos analisar se a função identidade é uma imersão. Ela é injetora, mas não satisfaz a versão mais forte da 2ª condição de homomorfismo. Afinal, o par $(2, 3) \in dom(A)$ não pertence à relação divide, mas o par $(h(2), h(3)) = (2, 3) \in dom(B)$ pertence à relação menor-ou-igual. Mais que isso, a função identidade não é sequer um homomorfismo de $A$ para $B$, pois não preserva destaques.
                \end{enumerate}
                Desse modo, $A$ não é subestrutura de $B$.
            \end{enumerate}
        \end{resolution}
    \end{question}
\end{rexercises}

\begin{exercises}
    \begin{question}
        Para cada sentença abaixo, construa uma assinatura e uma estrutura apropriada para representá-la, e, com base em uma interpretação, expresse-a na lógica de predicados.
        \begin{enumerate}
            \item Sejam $n$, $a$ e $b$ inteiros. Se $n \neq 0$, $n|ab$ e $mdc(n, a) = 1$, então $n|b$. (\textbf{Lema de Euclides})
            \item Todo grafo planar é 4-colorível. (\textbf{Teorema das Quatro Cores})
            \item Para toda proposição, ou ela é verdadeira ou sua negação é verdadeira. (\textbf{Lei do Terceiro Excluído})
            \item Não existe três inteiros positivos $a$, $b$ e $c$ tal que $a^n + b^n = c^n$, onde $n$ é um inteiro maior que 2. (\textbf{Último Teorema de Fermat})
            \item Um inteiro $n > 1$ é primo se, e somente se, $(n - 1)! \equiv -1 (mod n)$. (\textbf{Teorema de Wilson})
            \item Sejam dois inteiros $a$ e $d$ tal que $d \neq 0$. Então existem dois inteiros $q$ e $r$ tal que $a = qd + r$ e $0 \leq r \leq |d|$. (\textbf{Algoritmo da Divisão}) 
        \end{enumerate}
    \end{question}

    \begin{question}
        Sejam $A$ e $B$ estruturas definidas abaixo.
        \begin{table}[h!]
            \centering
            \begin{tabular}{c c}
                A & B \\
                \begin{structure}
                    {$\mathbb{N}$}
                    {}
                    {menor-ou-igual$(-)$}
                    {}
                    {}
                \end{structure}
                &
                \begin{structure}
                    {$\mathbb{Z}$}
                    {}
                    {divide$(-)$}
                    {}
                    {}
                \end{structure}
            \end{tabular}
        \end{table}
        \begin{enumerate}
            \item Defina um homomorfismo de $A$ para $B$.
            \item Determine se $A$ é subestrutura de $B$.
            \item Defina um automorfismo de $B$ diferente da função identidade.
        \end{enumerate}
    \end{question}

    \begin{question}
        Sejam $A$ e $B$ estruturas definidas abaixo.
        \begin{center}
            \begin{tabular}{c c}
                A & B \\
                \begin{structure}
                    {$\mathbb{Z}$}
                    {}
                    {}
                    {}
                    {soma$(-,-)$}
                \end{structure}
                &
                \begin{structure}
                    {}
                    {$0$ $1$}
                    {}
                    {}
                    {soma-mod-2$(-,-)$} 
                \end{structure}
            \end{tabular}
        \end{center}

        Mostre que a função $h$, definida a seguir, é um homomorfismo de $A$ para $B$.
        \begin{center}
            \begin{tabular}{l}
            $h: dom(A) \mapsto dom(B)$ \\
            $h(x) = \begin{cases}
                0 & \text{se $x$ for par} \\
                1 & \text{caso contrário}
            \end{cases}$
            \end{tabular}
        \end{center}
    \end{question}

    \begin{question}
        Seja $A$ a estrutura abaixo.
        \begin{center}
            \begin{structure}
                {$\mathbb{R}$}
                {}
                {$\leq$$(-,-)$}
                {$0$\\$1$}
                {soma$(-,-)$}
            \end{structure}
        \end{center}
        Considere estruturas com os mesmos componentes de $A$, mas com os domínios abaixo. Determine se cada uma delas pode ser subestrutura de $A$.
        \begin{enumerate}
            \item Conjunto dos inteiros positivos
            \item Conjunto dos inteiros
            \item $\{0, 1, 2\}$
        \end{enumerate}
    \end{question}

    \begin{question}
        Dados o conjunto $X = \{2, 6\}$ e a estrutura $A$ abaixo, determine $\langle X \rangle_A$, ou seja, a menor subestrutura de $A$ que contém $X$ no domínio.
        \begin{center}
            \begin{structure}
                {}
                {$0$ $1$ $2$ $3$ $4$\\$5$ $6$ $7$ $8$ $9$}
                {divide$(-,-)$}
                {$0$ $4$ $5$}
                {quadrado-mod-10$(-)$\\soma-mod-5$(-,-)$}
            \end{structure}
        \end{center}
    \end{question}
\end{exercises}