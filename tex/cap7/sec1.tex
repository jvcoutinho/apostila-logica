\section{Método da Resolução}

A filosofia do método persiste: ser eficiente para certas entradas e eficiente em reconhecer essas entradas. Os conceitos como a regra da resolução e a forma normal conjuntiva continuam valendo, mas dessa vez, temos diferenças cruciais.
\subsection{O Problema da Unificação de Termos}
Se em uma cláusula $C_1$ há um literal $L$ e em outra cláusula $C_2$ há o literal $\neg L$, a regra da resolução determina que podemos criar uma cláusula nova -- o resolvente de $C_1$ e $C_2$, sendo a disjunção de todos os outros literais de $C_1$ e $C_2$. Porém, uma condição importante é que $L$ em $C_1$ seja idêntico a $L$ (de $\neg L$) em $C_2$.

Em lógica proposicional, esse problema é trivial pois avaliamos apenas variáveis. Mas em primeira ordem, caso dois literais não sejam idênticos, devemos ser capazes de detectar se eles \textbf{podem} ser idênticos, utilizando de substituição de variáveis. Por exemplo, podemos aplicar a regra em $R(a)$ e $\neg R(x)$ se tomarmos $[a\diagup x]$, mas o mesmo não é possível com $P(a)$ e $\neg P(b)$, pois não há substituição que torne os literais idênticos. De modo geral, precisamos de um procedimento que determine se dois literais podem ser idênticos.

\begin{description}
    \item[Dados:] dois termos $t_1$ e $t_2$;
    \item[Pergunta-se:] existe uma substituição de variáveis que torne $t_1$ e $t_2$ idênticos? 
\end{description}

Caso tal substituição exista, dizemos que $t_1$ e $t_2$ são termos \textbf{unificáveis} e a substituição é dita \textbf{unificadora}. Em 1930, o matemático Jacques Herbrand definiu um algoritmo simples que resolve o problema usando regras de transformação de um conjunto de equações. Uma \textbf{equação} é um par de termos $(s = t)$. Um \textbf{sistema de equações} $S$ é um multiconjunto de equações e uma substituição $\Theta$ é unificadora de $S$ se ela unifica todas as equações de $S$. O conjunto de todas as substituições unificadoras de $S$ é denotado por $U(S)$. Esse conjunto pode conter várias substituições possíveis, mas uma delas é a ``ótima''. Por exemplo, para os termos abaixo:
\[f(y))\quad\text{e}\quad f(g(z))\]
A substituição $\Theta_1 = [g(b)\diagup y, b\diagup z]$ unifica os termos, mas não é a única possível. $\Theta_2 = [g(z)\diagup y]$ também os unifica e é mais simples. A substituição mais simples que unifica dois termos dentre as possíveis é dita \textbf{unificadora mais geral}.

O algoritmo de Herbrand usa como base o conceito de \textbf{forma resolvida}: uma equação está na forma resolvida em um sistema $S$ se for da forma $v = t$ (variável = termo) e $v$ for uma variável resolvida, ou seja, $v$ não ocorre em $t$ e em nenhuma outra equação de $S$. Caso todas as equações de $S$ estejam na forma resolvida, o sistema é unificável e o algoritmo devolve a substituição unificadora mais geral de $S$.

\begin{definition}{Regras de Transformação}
    \begin{description}
        \item[Eliminação de Equações Triviais] $S \cup \{t \? t\} \Rightarrow S$
        \item[Decomposição de Termos] $S \cup \{f(t_1,...,t_n) \? f(s_1,...,s_n)\} \Rightarrow S \cup \{t_1 \? s_1,...,t_n \? s_n\}$
        \begin{itemize}
            \item $S \cup \{f(t_1,...,t_n) \? g(t_1,...,t_m)\}$, onde $f \neq g$ ou $n \neq m \Rightarrow$ \textit{falha}.
        \end{itemize}  
        \item[Eliminação de Variáveis] $S \cup \{x \? t\} \Rightarrow S[x\diagup t] \cup \{x \? t\}$, se $x$ não ocorre em $t$.
        \begin{itemize}
            \item Se $x$ ocorre em $t \Rightarrow$ \textit{falha}.
        \end{itemize}
    \end{description}
    \tcbsubtitle{Algoritmo de Herbrand}
    \begin{description}
        \item[Entrada:] um sistema de equações $S$;
        \item[Saída:] Se $S$ for unificável, a unificadora mais geral de $S$; caso contrário, $S$ não é unificável.
    \end{description}
    \begin{enumerate}
        \item Para cada equação $s$ de $S$:
        \begin{enumerate}
            \item [1.1] Se $s$ estiver na forma resolvida, passe para a próxima.
            \item [1.2] Caso contrário, aplique uma das regras de transformação, com prioridade decrescente de cima pra baixo, e vá para o início.
            \item [1.3] Se não for possível aplicar uma regra, retorne \textbf{não é unificável}.
        \end{enumerate}
        \item Caso todas as equações de $s$ estejam na forma resolvida, retorne a \textbf{unificadora mais geral} formada pelas equações presentes em $S$.
    \end{enumerate}
\end{definition}

Para ilustrar, vamos executar o algoritmo sobre o seguinte conjunto de equações $S$:
\[S = \{f(g(z), x) = f(y, x), f(y,x) = f(y,h(a)), f(g(z), x) = f(y, h(a))\}\]
\begin{center}
    \begin{tabular}{r l}
        \textbf{DT} & $\{g(z) \? y, x \? x, f(y,x) \? f(y,h(a)), f(g(z), x) \? f(y, h(a))\}$ \\
        \textbf{EV} & $\{\mathbf{y \? g(z)}, x \? x, f({\color{Orange}g(z)},x) \? f({\color{Orange}g(z)},h(a)), f(g(z), x) \? f({\color{Orange}g(z)}, h(a))\}$ \\
        \textbf{EET} & $\{\mathbf{y \? g(z)}, f(g(z),x) \? f(g(z),h(a)), f(g(z), x) \? f(g(z), h(a))\}$ \\
        \textbf{DT} & $\{\mathbf{y \? g(z)}, g(z) \? g(z), x \? h(a), f(g(z), x) \? f(g(z), h(a))\}$ \\
        \textbf{EET} & $\{\mathbf{y \? g(z)}, x \? h(a), f(g(z), x) \? f(g(z), h(a))\}$ \\
        \textbf{EV} & $\{\mathbf{y \? g(z)}, \mathbf{x \? h(a)}, f(g(z), {\color{Orange}h(a)}) \? f(g(z), h(a))\}$ \\
        \textbf{DT} & $\{\mathbf{y \? g(z)}, \mathbf{x \? h(a)}, g(z) \? g(z), h(a) \? h(a)\}$ \\
        \textbf{EET} & $\{\mathbf{y \? g(z)}, \mathbf{x \? h(a)}, h(a) \? h(a)\}$ \\
        \textbf{EET} & $\{\mathbf{y \? g(z)}, \mathbf{x \? h(a)}\}$ 
    \end{tabular}
\end{center}
$S$ é unificável e a unificadora mais geral de $S$ é $\Theta = [g(z)\diagup y, h(a) \diagup x]$. Isso significa que, para os termos serem idênticos, basta substituir $x$ por $h(a)$ e $y$ por $g(z)$. Vejamos agora um exemplo onde o algoritmo falha:
\[S' = \{g(f(x,x)) = g(f(h(a), g(b)))\}\]
\begin{center}
    \begin{tabular}{l l}
        \textbf{DT} & $\{f(x,x) \? f(h(a), g(b))\}$ \\
        \textbf{DT} & $\{x \? h(a), x \? g(b)\}$ \\
        \textbf{EV} & $\{\mathbf{x \? h(a)}, {\color{Orange}h(a)} \? g(b)\}$ \\
        & $\{\mathbf{x \? h(a)}, {\color{Red}h(a) \? g(b)}\}$
    \end{tabular}
\end{center}

Uma vez que $h \neq g$, o algoritmo falha ao tentar aplicar uma regra em $h(a) \? g(b)$. Desse modo, o sistema não é unificável. De fato: não existe nenhuma substituição de variáveis que torne $x = h(a)$ e $x = g(b)$ ao mesmo tempo. 

No contexto de resolução, portanto, usaremos o algoritmo de Herbrand sempre que tentarmos aplicar a regra da resolução em dois literais complementares. Caso sejam unificáveis, procedemos com o resolvente e \textbf{aplicamos a unificadora mais geral neste}. Caso contrário, não aplicaremos a regra.

\subsection{Remoção dos quantificadores}
Já concluímos um dos problemas referentes ao método da resolução, que se preocupava com a identicidade de literais. Agora, veremos o outro grande problema, sobre a sintaxe da entrada. As entradas do método da resolução devem estar na forma normal conjuntiva, que, como sabemos, é uma conjunção de cláusulas. Porém, as fórmulas da lógica de primeira ordem podem possuir quantificadores, que não estão presentes nessa definição. Dessa maneira, precisamos removê-los. A estratégia que usaremos será a seguinte: colocar todos os quantificadores no início da fórmula, e então removê-los.

\subsubsection{Forma Normal Prenex}
Uma fórmula $\varphi$ está na \textbf{forma normal prenex} (\textit{prefixed normal expression}) se, e somente se, ela tem o seguinte formato:
\[(Q_1 x_1Q_2 x_2...Q_n x_n)(\psi)\]
Onde $(Q_1 x_1Q_2 x_2...Q_n x_n)$ é uma parte com quantificadores e chamada de \textbf{prefixo} de $\varphi$ e $\psi$ é uma fórmula sem quantificadores chamada de \textbf{matriz} de $\varphi$.

\begin{theorem}{}
    Para toda fórmula $\varphi$, existe uma fórmula $\psi$ tal que:
    \begin{itemize}
        \item $\psi$ está na forma normal prenex;
        \item $\psi$ é logicamente equivalente a $\varphi$.
    \end{itemize}
    \tcbsubtitle{Prova (transformação para a FNP)}
    Para provar o teorema, faremos uma prova por construção mostrando um método que transforma qualquer fórmula $\varphi$ para outra $\psi$ logicamente equivalente e na forma normal prenex.
    \begin{enumerate}
        \item Elimine os conectivos $\rightarrow$ e $\leftrightarrow$.
        \item Aplique repetidamente, além das equivalências conhecidas $(\neg \neg L) \equiv L$ (lei da dupla negação), as leis de De Morgan e a propriedade distributiva, as equivalências a seguir, até que a fórmula esteja na forma normal prenex.
        \begin{itemize}
            \item $\neg \forall x (\omega) \equiv \exists x (\neg \omega)$
            \item $\neg \exists x (\omega) \equiv \forall x (\neg \omega)$
            \item $\forall x (\varphi(x)) \land \forall x (\psi(x)) \equiv \forall x (\varphi(x) \land \psi(x))$
            \item $\exists x (\varphi(x)) \lor \exists x (\psi(x)) \equiv \exists x (\varphi(x) \lor \psi(x))$
            \item $Q x (\varphi(x))\equiv Q y(\varphi(y))$, onde $Q \in \{\forall, \exists\}$
            \item $Q x (\varphi(x)) \land \psi \equiv Q x (\varphi(x) \land \psi)$, onde $Q \in \{\forall, \exists\}$ 
            \item $Q x (\varphi(x)) \lor \psi \equiv Q x (\varphi(x) \lor \psi)$, onde $Q \in \{\forall, \exists\}$ 
        \end{itemize}
    \end{enumerate}
\end{theorem}

\subsubsection{Forma Padrão de Skolem}

Uma fórmula $\varphi$ na forma normal prenex está na \textbf{forma padrão de Skolem} se, e somente se, o prefixo de $\varphi$ não contém quantificadores existenciais. Em 1920, o lógico-matemático Thoralf Skolem definiu um método para remoção de quantificadores existenciais, que acabou ganhando seu nome: o \textbf{método da skolemização}.

Suponha a seguinte estrutura $A$:
\begin{center}
    \begin{structure}
        {$\mathbb{N}$}
        {}
        {$<$$(-,-)$}
        {}
        {}    
    \end{structure}
\end{center}


A sentença $\varphi = \forall x \exists y R(x,y)$ é verdadeira em $A$, pois para todo natural $x$ existe um outro maior que ele, por exemplo, seu sucessor. Dessa forma, podemos retirar o existencial, substituindo a ocorrência de sua variável pela função sucessor aplicada a $x$, ou seja, $\varphi' = \forall x R(x,f(x))$.
\begin{center}
    \begin{structure}
        {$\mathbb{N}$}
        {}
        {$<$$(-,-)$}
        {}
        {sucessor$(-,-)$}    
    \end{structure}
\end{center}

$\varphi'$ é verdadeira nessa nova estrutura que é idêntica a $A$, adicionando-se a função $f$. A skolemização é, portanto, o processo de remover quantificadores existenciais, substituindo as ocorrências das variáveis desses quantificadores por funções aplicadas a variáveis universalmente quantificadas anteriores a esses existenciais.

\begin{theorem}{Teorema de Löwenheim-Skolem}
    Seja $\varphi$ uma fórmula na forma normal prenex sobre uma assinatura $L$. Seja $\psi$ a fórmula resultante da remoção de cada quantificador existencial que ocorre em $\varphi$ e cujas variáveis correspondentes são substituídas por um termo do tipo $f(x_1,...,x_n)$, onde $f$ é um novo símbolo de função e $x_1,...,x_n$ são variáveis universalmente quantificadas imediatamente anteriores a esse existencial. Então, se existe uma $L$-Estrutura $A$ que é modelo para $\varphi$, é possível construir uma $L'$-Estrutura $B$ que é modelo para $\psi$, acrescentando a $A$ uma interpretação para cada símbolo novo de função em $L'$.
\end{theorem}
Caso não haja quantificadores universais anteriores a um existencial, a variável correspondente deste é substituída por um novo símbolo de constante (função de aridade zero). Uma função nova adicionada na assinatura é chamada \textbf{função de Skolem}, e uma constante nova é chamada \textbf{constante de Skolem}. A skolemização não preserva a equivalência das fórmulas resultante e original, mas preserva a satisfatibilidade: a original é satisfatível se, e somente se, a resultante é satisfatível.

Para ilustrar o processo de remoção de quantificadores, vejamos as seguintes fórmulas:
\begin{itemize}
    \item $\forall xP(x) \rightarrow \exists x S(x)$: \\
    $\equiv \neg \forall xP(x) \lor \exists x S(x)$ \\
    $\equiv \exists x (\neg P(x)) \lor \exists xS(x)$ \\
    $\equiv \underline{\exists x}(\neg P(x) \lor S(x))$ \littlelabel{FNP} \\
    $\exists x \Rightarrow$ adicionando uma constante de Skolem $a$ na assinatura; \\
    $\equiv \neg P(a) \lor S(a)$ \littlelabel{FPS}
    \item $\exists x(R(x,y) \rightarrow \forall y(P(z,y) \land \exists w(S(w,u) \lor \neg R(w,y))))$: \\
    $\equiv \exists x(\neg R(x,y) \lor \forall y(P(z,y) \land \exists w(S(w,u) \lor \neg R(w,y))))$ \\
    $\equiv \exists x(\neg R(x,y) \lor \forall m(P(z,m) \land \exists w(S(w,u) \lor \neg R(w,m))))$ \\
    $\equiv \exists x\forall m(\neg R(x,y) \lor (P(z,m) \land \exists w(S(w,u) \lor \neg R(w,m))))$ \\
    $\equiv \exists x\forall m(\neg R(x,y) \lor \exists w(P(z,m) \land (S(w,u) \lor \neg R(w,m))))$ \\
    $\equiv \underline{\exists x}\forall m\underline{\exists w}(\neg R(x,y) \lor (P(z,m) \land (S(w,u) \lor \neg R(w,m))))$ \littlelabel{FNP} \\
    $\exists x \Rightarrow$ adicionando uma constante de Skolem $a$ na assinatura; \\
    $\exists w \Rightarrow$ adicionando uma função unária de Skolem $f(-)$ na assinatura; \\
    $\equiv \forall m(\neg R(a,y) \lor (P(z,m) \land (S(f(m),u) \lor \neg R(f(m),m))))$ \littlelabel{FPS}
\end{itemize}

Assim, para colocar as fórmulas na forma normal conjuntiva, primeiro colocamos-as na forma padrão de Skolem e simplesmente ignoramos os quantificadores universais.