\begin{rexercises}
    % \begin{question}
    %     O termo $f(y,z)$ está livre para entrar no lugar da variável $y$ na fórmula\\ $\forall x(R(y, x) \rightarrow \forall z S(z, y))$? E na fórmula $\forall x(R(y, x) \rightarrow \forall y S(z, y))$?
    %     \begin{resolution}
    %         Observe que o enunciado é equivalente a perguntar se as substituições $[f(y,z)\diagup y]$ nas fórmulas $\forall x(R(y, x) \rightarrow \forall z S(z, y))$ e $\forall x(R(y, x) \rightarrow \forall y S(z, y))$ são possíveis de serem realizadas seguramente. 
            
    %         Para a primeira fórmula, vemos rapidamente que não, pois a variável $z$ em $f(y,z)$ se tornaria ligada, o que não deve acontecer. Para a segunda fórmula, a única ocorrência da variável $y$ que será substituída é a de $R(y,x)$, pois a de $S(z,y)$ é uma ocorrência ligada. Desse modo, as variáveis de $f(y,z)$ não se tornariam ligadas e, portanto, a substituição é possível.
    %         Podemos também responder ambas as questões aplicando a função recursiva.
    %     \end{resolution}
    % \end{question}

    \begin{question}
        Seja $\sigma = \exists x \forall y(R(x,y) \lor R(y, x))$. Defina duas estruturas $A$ e $B$ tal que $A$ seja modelo de $\sigma$ e $B$ seja contramodelo de $\sigma$.
        \begin{resolution}
            Queremos construir uma estrutura $A$ que satisfaça $\sigma$ sob alguma interpretação. Temos:
            
            \begin{center}
                \begin{structure}
                    {$\mathbb{N}$}
                    {}
                    {divide$(-,-)$}
                    {}
                    {}
                    {}    
                \end{structure}
            \end{center}
           
            $A$ é modelo para $\sigma$ pois, ao tomarmos $R$ como divide$(-,-)$, temos que 1 é um natural que divide todos os outros, tornando a fórmula $R(x,y)$ verdadeira para qualquer valor de $y$, e, por consequência, a disjunção.

            Agora, queremos construir uma estrutura $B$ que refute $\sigma$ sob alguma interpretação. Isso equivale a dizer que $B$ satisfaz $\neg \sigma \equiv \forall x\exists y(\neg R(x, y) \land \neg R(y, x))$ sob alguma interpretação. Temos:

            \begin{center}
                \begin{structure}
                    {$\mathbb{N}$}
                    {}
                    {menor$(-,-)$}
                    {}
                    {}
                    {}    
                \end{structure}
            \end{center}

            $B$ é modelo para $\neg \sigma$ (e, portanto, contramodelo para $\sigma$) pois, ao tomarmos $R$ como menor$(-,-)$, para todo natural $x$, podemos tomar $y$ como $x$. Desse modo, nem $x$ é menor que $y$ e nem $y$ é menor que $x$, satisfazendo a conjunção.
        \end{resolution}
    \end{question}

    \begin{question}
        Esboce o diagrama positivo da estrutura $A$ abaixo.
        
        \begin{center}
            \begin{structure}
                {}
                {$0$ $1$ $2$ $3$}
                {$<$$(-,-)$}
                {$1$}
                {dobro-mod-4$(-)$}
            \end{structure}
        \end{center}
        \begin{resolution}
            Queremos construir um esboço do conjunto que contém todas as sentenças atômicas que são verdadeiras em $A$ sob alguma interpretação. Primeiramente, notemos que $\langle \emptyset \rangle_A \neq A$, já que $3$ não é alcançável pelos destaques e funções de $A$. Desse modo, adicionamos $3$ aos destaques, assim como mais um símbolo de destaque na assinatura:
            \begin{itemize}
                \item 2 símbolos de destaque: $a$, $b$
                \item 1 símbolo de relação binária: $M$
                \item 1 símbolo de função unária: $d$
            \end{itemize}
            E então, escolhemos uma interpretação:
            \begin{itemize}
                \item $a^A = 1$; $b^A = 3$
                \item $M^A =$ $<$$(-,-)$
                \item $d^A =$ dobro-mod-4$(-)$ 
            \end{itemize}
            Assim, temos:

            $diag^+(A) = \{M(a, b), M(a, d(a)), M(d(a), b), M(d(d(a)), a), M(d(d(a)), d(a)),\\ M(d(d(a)), b), d(a) = d(b), d(d(a)) = d(d(d(a))), ...\}$
        \end{resolution}
    \end{question}

    \begin{question}
        Construa o modelo canônico do conjunto de sentenças atômicas $T$ a seguir.
        \begin{center}
            $T = \{R(a, b), f(a) = g(b), a = f(a), b = f(b), g(a) = f(a)\}$
        \end{center}
        \begin{resolution}
            Queremos construir a estrutura mais genérica possível que seja modelo para todas as quatro sentenças de $T$. Primeiramente, devemos identificar as classes de equivalência da relação $\sim$:
            \begin{itemize}
                \item Notemos que $f(a) = g(b)$, $f(a) = a$ e $g(a) = f(a)$. Desse modo, $a$, $f(a)$, $g(b)$ e $g(a)$ constituem uma classe de equivalência, e tomaremos $a^\sim$ como sua representante.
                \item Como $b = f(b)$, $b$ e $f(b)$ constituem outra classe de equivalência, e tomaremos $b^\sim$ como sua representante. 
            \end{itemize}
            Assim, temos a estrutura $D$:

            \begin{center}
                \begin{structure}
                    {}
                    {$a^\sim$ $b^\sim$}
                    {$R(-,-)$}
                    {$a^\sim$ $b^\sim$}
                    {$f(-)$ $g(-)$}
                \end{structure}
            \end{center}

            Onde:
            \begin{itemize}
                \item [] $R^D = \{(a^\sim, b^\sim)\}$
                \item [] $f^D(a^\sim) = a^\sim$, $f^D(b^\sim) = b^\sim$, $g^D(a^\sim) = a^\sim$ e $g^D(b^\sim) = a^\sim$
            \end{itemize}
        \end{resolution}
    \end{question}
\end{rexercises}

\begin{exercises}
    \begin{question}
        Seja $A$ a estrutura abaixo:
        \begin{center}
            \begin{structure}
                {$\mathbb{N}$}
                {}
                {menor$(-,-)$}
                {$0$}
                {soma$(-,-)$\\produto$(-,-)$}
            \end{structure}
        \end{center}
        Determine se $A$ é modelo ou contramodelo (ou ambos) para as sentenças a seguir. Considere que $a$, $R$ e $f$ são símbolos da assinatura de $A$.
        \begin{enumerate}
            \item $\forall x\exists y (f(x, y) = a)$
            \item $\forall x\forall y(f(x, y) = f(y, x))$
            \item $\forall x\forall y(R(x, y) \rightarrow \exists p(\neg (p = a) \land f(x, p) = y))$
            \item $\forall x\forall y(\exists z (f(x, z) = y) \rightarrow R(x, y))$
        \end{enumerate} 
    \end{question}

    \begin{question}
        Seja $B$ a estrutura abaixo:
        \begin{center}
            \begin{structure}
                {Cadeias binárias de 4 bits}
                {}
                {$\geq$$(-,-)$}
                {$0101$\\$1010$}
                {AND$(-,-)$\\XOR$(-,-)$}
            \end{structure}
        \end{center}
        Defina uma assinatura $L$ para $A$, e então escreva uma sentença sobre $L$ para a qual $A$ é tanto modelo quanto contramodelo.
    \end{question}

    \begin{question}
        Construa o modelo canônico dos seguintes conjuntos de sentenças atômicas:
        \begin{enumerate}
            \item $\{f(a) = h(b, c), f(a) = f(c), b = g(a, b), R(a), R(c), S(b, c)\}$
            \item $\{R(a), M(f(b), b), f(a) = f(b), f(b) = a, h(f(a), a) = b, h(a, b) = f(f(a)), h(b, a) = h(h(f(a), a), b), R(h(f(h(a, b)), b)), M(a, a)\}$
        \end{enumerate}
    \end{question}

    \begin{question}
        Para cada estrutura a seguir, defina uma assinatura e uma interpretação da assinatura na estrutura, e construa um esboço do seu diagrama positivo.
        \begin{enumerate}
            \item 
                \begin{structure}
                    {}
                    {$0$ $1$ $2$ $3$}
                    {ímpar$(-)$\\maior$(-, -)$}
                    {$3$}
                    {subtração-mod-4$(-, -)$\\cubo-mod-4$(-)$}
                \end{structure}
            \item
                \begin{structure}
                    {$\mathbb{N}$}
                    {}
                    {primo$(-)$}
                    {$1$}
                    {sucessor$(-)$\\quadrado$(-)$}
                \end{structure}
        \end{enumerate}
    \end{question}
\end{exercises}